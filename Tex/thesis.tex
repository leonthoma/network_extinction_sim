\documentclass[12pt,a4paper]{article}
\usepackage[utf8]{inputenc}
\usepackage{csquotes}
\usepackage[english]{babel}
\usepackage[T1]{fontenc}
\usepackage{graphicx}
\usepackage{float}
\author{Leon Thoma}
\title{Master thesis in environmental sciences}
\usepackage[backend=biber, style=authoryear, giveninits=true]{biblatex}%  !!!In bib only state the first char of the first name!!!
\AtBeginBibliography{\footnotesize}
\usepackage{url}
\usepackage[font={footnotesize}, labelfont=bf]{caption}
\usepackage[table,xcdraw]{xcolor}
\usepackage{caption}
\usepackage{subcaption}
\usepackage{amssymb}
\usepackage{amsmath}

\addbibresource{extc_net.bib}
\defbibheading{none}{}
\usepackage[colorlinks]{hyperref}
\hypersetup{
    colorlinks,
    linkcolor={black!50!black},
    citecolor={black!50!black},
    urlcolor={blue!80!black}
}
\usepackage{multicol}
\usepackage[margin=1in]{geometry}
\usepackage{tabularx,booktabs}
\usepackage{lscape}
\usepackage{rotating}

\begin{document}
\begin{titlepage}
	\centering
	\includegraphics[width=0.25\textwidth]{logo-grundversion-01}\par\vspace{1cm}
	{\scshape\large Albert-Ludwigs-Universität Freiburg\par}
	\vspace{1.25cm}
	{\scshape\large Thesis for the attainment of the degree:\\ Master of science\par}
	\vspace{.75cm}
	{\scshape\large Faculty for Environment and Natural Resources\par}
	\vspace{.75cm}
	{\scshape\large Chair of Biometry and Ecosystem Analysis\par}
	\vspace{.75cm}
	{\Large\itshape Rewiring networks under species loss: a simulation
study
\par}
	\vspace{.75cm}
	{\Large\itshape Leon Thoma\par}
	\vspace{.25cm}
	{\scshape\normalsize E-mail: leonthoma@gmx.de\par}
	\vspace{.15cm}
	{\scshape\normalsize Matriculation number: 3733253\par}
	\vspace{.75cm}
	\large Examiner:\par
	\large Prof. Dr. Carsten  \textsc{Dormann} \par
	\vspace{.5cm}
	\large Co-Examiner:\par
	\large Dr. Anne-Christine \textsc{Mupepele}
	\vfill

	{\large \today\par}
\end{titlepage}
	\tableofcontents
	\newpage	
	\section{Abstract} 

\newpage%
%
% RECHECKING:
% GO THROUGH ALL CITATIONS IN JABREF FOR CORRECT SPECIAL CHARACTERS AND GENERAL CHECK
%
\section{Introduction}
% Content
% Short intro to networks
% Drivers of species interaction
	% Neutral theory
	% Niche theory
	% Spatio-temporal variability
% Intro to extinction simulation
	% What has been done
	% State of knowledge
%	-> connection to gap of knowledge
% Gap of knowledge/Approach
	% Why ?
	% Reserch questions





Contrary to a main question in the field of invasion ecology, how novel species influence interactions among species, stands the question of how species leaving an ecosystem affects the remaining species. 

% species leaving network can have influence on connection between sp over different biomes/ ecosystems (terrestrial/aquatic; knight 2005 in colwell 2012)
% coextinction is difficult to document (colwell 2012):
	% determining host specifity -> interaction partners might become extinct before their hosts (Moir 2010, 2012b, Powell 2011)
	% if spcies are reintroduced to wild, interactions partners might be affected by treatment used to bring up reintroduced species (e.g. biocides)
	% 

%
%The introduction of species was studied in a wide variety of ecosystems (references here) and across many species (references here), studying the effects of extinctions, however is much harder. This is due to the fact that  
% Intro to topic:
	% network ecology
		% lau2017
		% poisot2016; 10.1111/1365-2435.12799
		% vazquez2022
\subsection{Network ecology}
Network ecology has helped researchers to develop a more detailed understanding of ecological communities. From analysing structures within networks and identifying patterns \parencite{Jordano1987, Dunne2002}, % Olesen2007, Bascompte2003 
to finding the underlying processes that lead to the observed patterns \parencite{Rezende2007, Vazquez2009, Thebault2010}.

% relation between network properties/structure  https://doi.org/10.1111/j.1365-2656.2010.01688.x 
%Ecological communities are subject to constant change, due to a variety of reasons such as habitat fragmentation, global warming, introduction of species or loss thereof, to name some \parencite{}. 

Focused on mutualistic interactions such as plant-pollinator networks since these are very well studied.

\subsubsection{Species interactions}
A major part of understanding the formation, upkeep and remodelling that is observed in ecological communities, is how interactions between species are formed as well as how and why the change. Historically, researchers tried to explain interactions with two main theories. Niche theory based the reasoning on biological information such as morphology, phenology, behaviour, evolutionary history and phylogeny \parencite{Webb2002, Jordano2003, Vazquez2009}. % all from vazquez2022
Consequently, the probability of encounter is shaped by the traits of species.
% Some possible links are unrealized because of the mismatch in niche space (for example due to missing phenological overlap)\parencite{Canard2012}.
The other approach that was developed, called neutral theory, argues that species are equivalent in their ecological role and interactions are the result of random encounters. Therefore, interactions in communities should mostly be shaped by abundance \parencite{Vazquez2005, Vazquez2007}.\\
There are studies providing evidence that processes from both theories are underlying species interactions \parencite{Vazquez2009a, Verdu2011}. Further research has shown that the importance of the proposed drivers vary, for example that phenological overlap and trait matching are more important than abundance \parencite{Vizentin-Bugoni2014}, that native and exotic species are affected differently by niche processes \parencite{Peralta2020} or that they vary along a latitudinal gradient \parencite{Sonne2020}. Thus most studies focused on predicting interactions between species, combined processes of neutral and niche theory, most commonly including traits, phenology, phylogeny and abundance as variables \parencite{Pearse2013, Olito2015, Morente-Lopez2018, Kotula2021, Benadi}.

 \par 
More recently however, the effect of temporal and spatial variation on interactions and network structure moved into focus \parencite{Schwarz2020, CaraDonna2021}. \citeauthor{Olesen2008} showed that interactions varied in plant-pollinator networks due to temporal and spatial turnover \parencite{Olesen2008}. \citeauthor{CaraDonna2017} found that temporal dynamic in networks was high, and especially that species frequently changed interaction partners \parencite{CaraDonna2017}.

% for trait see bartomeus2018

% Not really interested in influence of choice method or removal sequence on robustness. rather, how do a, t, l influence networks, e.g. structures (h2', connectance, etc.), interactions or robustness. also if a ,t or l were used to determine rewiring partner what were the results ? 

In the face of the current rate of extinctions (e.g. see \cite{IPBES2019}), it is crucial to develop a deeper understanding of the changes an extinction event induces in ecological communities, in order to preserve it's function and prevent extinction cascades. 
\subsection{Extinction simulation}
	% Different approaches (Memmot, Kaiser-Bunbury, Vieira, Baumgartner, Vizentin-Bugoni)
The following 3 models mark cornerstones in advances of extinction cascade modelling. In 2004 \citeauthor{Memmott2004} simulated secondary extinction for plant-pollinator networks using three scenarios for initial extinctions of pollinators (but not plants). Random extinction (Null model), sequential removal of species starting with the most linked, and sequential removal starting with the least linked species. A species was considered to be coextinct if it lost all its interactions. However, the approach disregarded a few key points, such as the quality of a given interaction (i.e. pollination success and frequency of visits), the fact that plants can reproduce even without pollinators e.g. by selfing and arguably most important that species don't compensate for lost interaction partners (either by creating new interactions or enhancing existing ones). Nevertheless, the research of \citeauthor{Dunne2002} and \citeauthor{Memmott2004} provided helpful insights into network structure showing that species in less connected networks are more prone to suffer from coextinction. \par

A first quantitative approach to extinction modelling in mutualistic networks considering behavioural shifts (i.e. interaction rewiring) was done by \citeauthor{Kaiser-Bunbury2010} in 2010 \parencite{Kaiser-Bunbury2010}. They based rewiring on interactions between species that were observed at one point of time when the site was sampled. For example, if a species lost all it's interaction partners it would remain in the model with it's full interaction strength if a partner was present with which they observed an interaction in the network at an earlier or later point in time. Their results indicated that networks are more robust to extinction when rewiring is considered since it's 'artificially' increasing connectance. \par

In 2019 \citeauthor{Vizentin-Bugoni2019} extended the approach of \citeauthor{Memmott2004} and \citeauthor{Dunne2002} \parencite{Memmott2004, Dunne2002} for (co)extinction simulation.
% by enhancing the functions provided in the bipartite R-package \parencite{dormann2008}. 
%% Rephrase !!!
Rewiring was implemented as a two step process. First a potential partner is chosen based either on (1) variables known to be important for species interactions in the studied network (e.g. abundance) or (2) at random. Second, the model computes a rewiring probability for the chosen partner, accounting for species abundance, phenological overlap, morphological traits and combinations thereof. Finally, a binomial trail will determine if rewiring is successful or not. Furthermore, the model allows its users to choose the number of rewiring attempts \footnote{
One try with a single partner, 
multiple tries with a single partner, 
one try with each partner, 
multiple tries with multiple partners
}. Following the assumption of \citeauthor{Memmott2004} and \citeauthor{Dunne2002}, a species was considered coextinct if it lost all it's interaction partners.
\par


	% general results of extc sim w/ rewiring
		% robustness increases w/ rewiring: dunne2002, eklöf2006, thebault & fontaine2010, baumgartner2020
The results of research on extinctions and their propagation in ecological networks have shown, that loss of generalists causes a higher rate of coextinction than loss of specialists (\cite{Memmott2004, Kaiser-Bunbury2010, Traveset2017, Bastazini2018, Vizentin-Bugoni2019, Biella2020}; but see \cite{Dunne2002}). 
-> WHY

Another result was a substantially higher robustness in mutualistic bipartite networks, when species were removed from the higher trophic level compared to when lower trophic level species were removed \parencite{Schleuning2016}.

%This was the case irrespective of how the species that was to be removed was chosen, i.e. random, abundance based, etc.(RECHECK!!!) 
%This is most likely due to the fact, that species' roles in networks are highly asymmetric, meaning species belonging to the higher trophic level are very dependent on their interaction partners from the lower level, but not vice versa \parencite{Bascompte2006}.

\citeauthor{Dunne2002, Ekloef2006} and, \citeauthor{Thebault2010} found a positive correlation of network robustness and connectance \parencite{Dunne2002, Ekloef2006, Thebault2010}. However, \citeauthor{Vieira2015} showed that the opposite was the case \parencite{Vieira2015}. They argued, that the assumption, that coextinction only occurs when a species loses all its interaction partners, which the previous studies made, led to their finding. Recently, \citeauthor{Baumgartner2020}, challenged the reasoning of \citeauthor{Vieira2015} by showing a positive correlation between network robustness and connectance using an extinction algorithm that accounts for the interaction strengths (i.e. dependence), intrinsic demographic dependence, and dissimilarity of species to estimate coextinction susceptibility \parencite{Baumgartner2020}. 
% by including coext threshold this can be tested here !!! 


\parencite{Kaiser-Bunbury2010, Schleuning2016, Timoteo2016, Costa2018} higher robustness due to rewiring. % also (Ramos‐Jiliberto et al.,
%2012; Nuwagaba et al., 2017) from baumgartner2020; compare with blüthgen2010 and vieira2015


% Considering this...
% i will use....
% to find out that...

% Explaining problem/ gap of knowledge
Despite the insights previous studies provided, it is still unclear if robustness is influenced by how species interact (ie. what drives the interactions) and find new partners to substitute lost interactions with extinct partners. To provide further insight into this possible interrelationship, this study will focus on modelling extinction cascades using simulated networks controlling the influence of variables found to predict species interactions.\\

% For example, using the same rewiring criteria, does robustness change when the main driver of interactions in the network is changed ? Its still unknown if e.g. the abundance of species in a network changes the propagation of extinctions. 
	% rewiring methods
%	not clear how different rewiring mechanisms affect extinction propagation. especially in bugoni et al combining rewiring mechanisms led to lower effect size than single rewiring mechanisms

% Explain the approach
The analysis of this study was a two step process. First, networks were simulated and then used in the extinction algorithm to assess network robustnesses (see fig. \ref{fig:extc_alg}).  
Network simulations were based on the approach presented by \citeauthor{Benadi} \parencite{Benadi}. The 'observed' interactions in a network are simulated by combining abundance, trait, and phylogenetic information (also referred to as community variables). The influence of the three variables on the formation of interactions between species was altered by using either one, none or all of them as the main driver. Additionally, a group was added, where the importances of the three variables mentioned above were not changed. Thus, six ways of species interaction simulations were implemented.
To identify differences in the network structures among the species interaction scenarios, the two dimensional shannon entropy (H2') was calculated as a measure for specialization \parencite{Bluethgen2006}.



%	simulating networks has the advantage of equal "sampling effort" and is therefore not affected by sampling biases !!
	
%The constant 're-observation' of species interactions during the extinction simulations functions as a simple proxy of spatio-temporal variability of networks. 
% Research questions
%Does network robustness change for different rewiring methods when the driving force of interactions in networks is altered ?
%Is network robustness affected by underlying interrelations between the drivers of species interaction and their way of choosing rewiring partners 
% higher connectance leads to higher robustness (following dunne2002, eklöf2006, thebault & fontaine 2010, baumgartner2020
The questions this study aims to answer are:
\begin{enumerate}
\item Is network robustness influenced by the way species interactions are simulated ?
\item Does the choice of rewiring method lead to different network robustnesses ?
\end{enumerate}

% "third question"
%The interrelationship between species interaction scenarios and rewiring method was not part of the main questions of this study, however, since there seemed to be some interaction between the two, the relation of robustness and the percentage of species with only one interaction partner among rewiring methods was compared.

For the first question only simulations where no rewiring was used were considered. This allowed for unbiased comparisons between the species interaction scenarios. In order to compare rewiring methods, the extinction simulations were run only with networks where community variables were left unchanged.



%first question: what influence does com_vars have on robustness (compare all com_vars using norew)
%	-> 
%second question: what influence does rew have on robustness (compare all rew using org networks) -> focus of thesis should be rew methods !
%	-> rew should be most efficient for specialised networks
%	
%resulting third question: is there an interrelationship of com_vars and rew that influences robustness (use no of singletons per network comparing rew methods)
%	-> comparing H2' would expect highest effect of rewiring for specialised networks, but maybe only highly abundant/trait matching/phylogeny matching species are specialised and the rest isn't. therefore check no of singeltons in networks (ie sp most endangered by coextinction) and compare the robustness 



	\section{Material \& Methods}
	The simulations in this study were executed in R \parencite{Rcore} (version 3.6.3) based on the functions of the tapnet\footnote{The paper is yet to be published, but the package and it's vignette are available via CRAN and GitHub; https://cran.r-project.org/web/packages/tapnet/index.html, https://github.com/biometry/tapnet} \parencite{Benadi} and bipartite \parencite{Dormann2008} packages as well as the approach of the extinction algorithm developed by \citeauthor{Vizentin-Bugoni2019} \parencite{Vizentin-Bugoni2019}. Some of the functions provided by these sources were modified or augmented to match the specific needs of the presented approach. The source code of this analysis is available as a supplementary file. A schematic overview of the network and extinction simulations of this study is presented in the figure below.	
	
	\begin{figure}[H]
	 \includegraphics[width = \textwidth]{/home/leon/Documents/Uni/M.sc/Master Thesis/Networks/approach_overview}
	 \caption{\textbf{Schematic overview of network \& extinction simulations} The left hand side shows the process of network simulation. From the initial network five networks were simulated according to the approach described in section \ref{sec:net_sim}. The right hand side shows the simplified structure of the extinction algorithm. The primary extinction (2.1) in each iteration is determined based on the user specified methods comprising trophic level (2.1.1) and species selection (2.1.2) (see section \ref{subsec:extc_alg} step \ref{itm:etxc} for details). Network A shows the species that is removed in black and the lost interactions with its partners in red. The algorithm then checks if other species are coextinct (2.2) by comparing the coextinction threshold to the amount of remaining interactions of each species (see section \ref{subsec:extc_alg} step \ref{itm:track} for details). Network B depicts the species lost due to coextinction in black with its interactions in orange. The next step is to determine interaction rewiring (2.3). Using one of the six rewiring methods (2.3.1 to 2.3.6), species try to rewire to new partners or strengthen interactions with their other partners (see section \ref{subsec:extc_alg} step \ref{itm:rew} for details). Network C shows possible rewiring for one species (shown in black) based on abundance (red), traits (blue) or phylogeny (beige).
	 The last step of each iteration is redrawing the interactions for species that rewired from a multinomial distribution with the updated interaction probabilities (see section \ref{sec:net_sim} for details). The algorithm stops redrawing networks once only two species in the trophic level chosen for primary extinction are left. The remaining species are determined by running steps 2.1 - 2.3.}
	 \label{fig:extc_alg}
\end{figure}
		\par


% describe general approach:
	% initial network sim
	% sim webs w/ com_var
	% choose rewiring method & extinction method
	% run extc models

	\subsection{Network simulation} \label{sec:net_sim}
To simulate the networks, each of the three community variables (abundance (a), traits (t), and phylogeny/latent traits (l)) were provided as an independent two-dimensional matrix defined by the species of the two trophic levels. The algorithm first derives interaction frequencies from the linear combination of the three community variable matrices, which are all scaled to sum of one before and after their multiplication. Interaction frequencies are then used as probabilities in a multinomial draw to calculate the desired number of 'observed' interactions (see also fig. 1 in \cite{Benadi} for a conceptual overview).	 The community variable matrices were the same for all networks and were inherited from an initial network simulation with the default tapnet algorithm. The number of species in the trophic levels of the initial network was 40 and 50 for the lower and higher level, respectively. Total number of observations was set to 1111. The number of simulated traits was set to 2 for phylogenetically correlated and uncorrelated traits each.\par
	% caradonna: nobs: 30000, ntraits: 2, nlow: , nhigh:
	% kaiser-bunbury: nobs: 1278, ntraits: NA, nlow: , nhigh:
	% costa: nobs: 	3974, ntraits: NA, nlow: , nhigh:
	% bugoni: nobs: na, ntraits; 2, nsims:1000, nlow: , nhigh
	% schleuning:   see supp mat.
	% memmot (10.1046/j.1461-0248.1999.00087.x): nobs: 2183, nlow: , nhigh: 
	% vazquez/simberloff (10.1046/j.1461-0248.2003.00534.x): nobs: 5285, nlow: , nhigh:
	

	% using simulated abundances, traits and other params (paramlist) simulate omat for all com importance scenarios
	Using the abundance \& trait matrices and the phylogenetic eigenmatrices from the initial simulation, 1000 networks were simulated for each of the six species interaction scenarios. This was achieved by exponentiating the matrices of community variables with either 0.1 or 1.9, to match their pattern to the interaction matrix\footnote{i.e. the result of the linear combination of community variable matrices}. For networks with abundance driven interactions, the abundance matrix was exponentiated with 0.1, the other two with 1.9. When interactions in a network should be driven mainly by traits, the abundance and trait matrices were exponentiated with 0.1 and the phylogeny matrix with 1.9. In networks where phylogeny dominated interactions, abundance and phylogeny matrices were exponentiated with 0.1, and the trait matrix with 1.9. When none of the community variables should be dominating interactions between species, all community variable matrices were exponentiated with 0.1. Conversely, if all community variables were set to have a higher influence on species interactions the exponent of all matrices was 1.9. For the initial networks none of the variables were regarded as especially important, matrices were thus left unchanged (i.e. the exponent was 1). The scenarios will be referred to as Atl, aTl, atL, atl, and ATL, with capital letter(s) indicating drivers of species interactions. The simulations of the initial network are called original.\par
	
	 The simulation of networks produced pairings where no interaction between species occurred (from now on called 'improbable interactions'). This is due to interaction probabilities being so small that an interaction is not 'observed' every time the network is simulated. Therefore, species pairs with improbable interactions were removed before using the networks for extinction simulation. However, since marginal totals were used to identify and delete species pairs with improbable interactions, there were cases where after removing a species, other pairings had no observed interactions. For example, if a species had only one interaction with another species that was deleted, the first species was then left without interactions. Because of this, the actual number of species per web was not equal to the number specified during network simulation. The distribution of improbable interactions for all networks can be seen in figure \ref{fig:dead_int}.

% Improbable interactions
\begin{figure}[H]
	 \includegraphics[width = \textwidth]{/home/leon/Documents/Uni/M.sc/Master Thesis/Networks/models/plot_sink/dead_pre_both_rev}
	 \caption{\textbf{Number of improbable interactions} The boxplots show the number of improbable interactions for each community variable importance scenario.
	 The left plot (A) shows species from the lower trophic level, the right plot (B) shows species from the higher trophic level. See section \ref{sec:net_sim} for a definition of improbable interactions and a detailed explanation of the abbreviations of community variable importance scenarios.}
	 \label{fig:dead_int}
\end{figure}

% by simulating networks problematic sampling effects are avoided (blüthgen2010, vazquez2022)

	\subsection{Extinction simulation} \label{sec:extc_sim}
	After networks were simulated, a modified version (see supplementary files 'extc.alg') of \citeauthor{Vizentin-Bugoni2019}'s algorithm was used to simulate extinction. The algorithm used here is structured as follows (see fig. \ref{fig:extc_alg} for a visual overview). 
%To easily distinguish the different scenarios, the following naming convention was established. The trophic level from which species are removed during the extinction simulation as well as the rewiring method are noted in superscript, respecitvely. For example Atl\textsubscript{lower, abund}, refers to a simulation where Abundance was the main driver of species interactions during the network simulation (see section \ref{sec:net_sim} for network abbreviations), species were removed from the lower trophic level during extinction simulation with abundance based rewiring.

\subsubsection{Extinction algorithm} \label{subsec:extc_alg}
\begin{enumerate} 
	\item \textbf{Delete improbable interactions} {\small Check if improbable interactions were introduced in the simulation process and delete them prior to extinction simulation.}
	\item \textbf{Set the adaptability of species} {\small This represents a probability which is used to express a species inclination to change interactions partners. The values range from 0 to 1 and are either default to 0.5 or originate from a uniform distribution.}
	\item \label{itm:etxc} \textbf{Extinction} {\small Delete a species from the network based on the provided methods (lower or higher trophic level).}
%This is executed using the extinction.mod function from \citeauthor{vizentin-bugoni2019} which is based on the extinction function from the bipartite package.}
%	\item \label{itm:failsafe} \textbf{Handling improbable interactions} {\small Since the algorithm redraws from a multinomial distribution using the updated interaction matrix to generate the network (see step \ref{itm:redraw} for details), improbable interactions can be introduced. This step is implemented to make sure the extinct species has valid interactions. If the extinct species has no interactions, the previous step is repeated until a species with valid interactions was found or three tries are reached. If this process is unsuccessful, the algorithm 'skips' the current extinction iteration by skipping steps (\ref{itm:rew}, \ref{itm:shift}, \ref{itm:track}, and \ref{itm:update}) and redraws a new network (i.e. continuing at step \ref{itm:redraw}). The simulation is stopped if there are three instances where iterations were skipped because of improbable interactions. This is an optional feature and users can decide if they want to use it or not.}
	\item \label{itm:rew} \textbf{Rewiring} {\small All species that had at least one observed interaction with the extinct species, will try rewiring. The choice of the rewiring partner, i.e. the way a species that the afore mentioned species will try to rewire to is determined, can be set by the user and comprises the following five options (all using the values from the initial simulation)}
		\begin{itemize}
		\item \label{itm:abund_rew} \textbf{Abundance} {\small The species with the highest abundance is selected.}
		\item \label{itm:trait_rew} \textbf{Traits} {\small Euclidean distances of all traits are calculated. The species with the smallest trait distance across all traits compared to the extinct species is selected.}
		\item \label{itm:phylo_rew} \textbf{Phylogeny} {\small The species with the lowest phylogenetic distance to the extinct species is chosen. If multiple species have the same distance, one is selected at random.}
		\item \label{itm:AT_rew} \textbf{Abundance x Trait} {\small The rewiring probabilities were calculated according to the respective method and their sum was used.}
		\item \label{itm:AP_rew} \textbf{Abundance x Phylogeny} {\small The rewiring probabilities were calculated according to the respective method and their sum was used.}
		\end{itemize}
	\item \label{itm:shift} \textbf{Preference shift} {\small Interaction probabilities are updated, simulating the process of species changing their foraging preferences due to losing one of their interaction partners. New interaction probabilities are the sum of (1) the values of the interaction between the species trying to rewire and the rewiring partner and (2) the values of the interaction between the extinct species and the species that try to rewire. The latter is multiplied with the adaptability value of each species to account for the differences in how well or how willingly they rewire. If no adaptability values are provided, the default vaules was arbitrarily set to 0.5}
	\item  \label{itm:track} \textbf{Tracking (co)extinctions} {\small Users can specify a threshold which determines coextinction of species based on the percentage of remaining interactions. If the quotient of total interactions after and before an extinction step is below or equal to the threshold, a species will be considered coextinct. Since the number of interactions are resimulated in each iteration, the divisor is the number of interactions of the last iteration and not the number of interactions of the web provided initially. Finally, the number of extinct and remaining species are noted for the higher and the lower trophic level.}
	\item \label{itm:update} \textbf{Update matrices} {\small The extinct species are deleted from the network and from the interaction matrix.}
	\item \label{itm:redraw} \textbf{Redraw interactions} {\small The updated interaction matrix is used to simulate the number of interactions between the species trying to rewire and the determined rewiring partner, by drawing from a multinomial distribution. The number of interactions simulated this way is divided by two (rounded down) and added to the original number of interactions between these species.}
	\end{enumerate}
	% running extinction functions. using modified version of bugoni's extinction fy
		% algorithm structure:
			% delete dead interactions from webs
			% set adaptability (prob) of sp; either draw from unif dist or .5 for all
			% delete sp from web using extinction.mod from bugoni (based on bipartite)
			% dead interaction handling; retry max three times to find sp that has also valid interactions, if unsuccessful skip iteration
			% choice of rewiring partner (i.e. sp that shall replace extc sp; always compared to extc sp):
				% abund: use abundances from init sim to select sp with highest abund
				% trait: calculate euclidean distances of all traits from init sim; select sp with smallest trait distance over all distances
				% phylo: use sp. with lowest phylogenetic distance; if more than 1 sp have same distance randomly choose one
			% shift interaction probs; sum of interaction probs of rew partner and extc sp and interaction probs of extc sp and sp that try rewiring (i.e. old 					  interactions partners of extc sp) times the adaptability of each sp
			% df with number of extc sp & remaining sp per level is updated
			% update imat; retain dead interactions that were introduced by imat in previous step
			% calculate new web from updated interactions; check if dead interactions were introduced
			
		% following extinctions were run (always for lower and higher level extc, except for 'both'; also w/ and w/o rewiring):
			% org = default com importance; random extinction of sp
			% abund = abundance led extinction of sp
			% both = random extc from lower or higher level, random extc of sp
			% random = random extc of sp
				% partner choice for rewiring based on values from init sim; using cophenetic.phylo fy for phylogenetic distance
Partner choice in the rewiring process of the extinction simulations was based on the abundances, traits and phylogeny of the initial simulation. For phylogeny, the phylogenetic distances of the higher and lower trophic level were computed. Two species removal scenarios were simulated with 10 runs for each network. (1) removing species from the lower trophic level and (2) removing species from the higher trophic level. The coextinction threshold was set to 0 \%, i.e. a species was coextinct when it lost all interaction partners and adaptability of all species was set to 0.5.
%\begin{table}[H]
%\centering
%\caption{\textbf{Overview of extinction models} If community variables were set to low or high contribution to determine species interactions (see \ref{sec:net_sim} for details) one model with each setting was simulated. For trophic levels either the higher or lower level was chosen and extinctions were simulated. In the 'Trophic' model either the lower or higher trophic level was chosen randomly each extinction step and a species was removed. Except for the 'Abundance' model, the species that was removed was chosen at random. For the 'Abundance' model the species with the lowest abundance was removed sequentially. Each model was run with and without rewiring.}
%\label{tab:models}
%\resizebox{\textwidth}{!}{%
%\begin{tabular}{llll}
%Model & Community variables & Trophic levels & Species removed \\ \hline
%Null & None & lower; higher & Random \\
%Base & Abundance; Traits; Phylogeny & lower; higher & Random \\
%Trophic & Abundance; Traits; Phylogeny & lower \& higher & Random \\
%Abundance & Abundance; Traits; Phylogeny & lower; higher & Abundance based
%\end{tabular}%
%}
%\end{table}


Following the extinction simulations, the robustness of each network was calculated. To assess the effects of rewiring and network simulation on robustness a ANOVA was used with the number of species from the lower and the upper trophic level, community variable importance, rewiring method, $H2'$, the interaction between rewiring method and $H2'$, as well as the extinction scenario \footnote{i.e. removing species from the lower or from the higher trophic level} as predictors.
%Additional ANOVAs were used to see how the effects changed when the data was partitioned by species removal, network simulation scenarios, and rewiring methods.

 
%	% computing means of all simulations for each scenario
%	The individual runs of each extinction scenario led to different lengths of the extinction sequences, therefore the longest sequence of each of the 10 simulations per web was chosen and the simulations with shorter sequences were filled with zeros to match their lengths. Then the mean per web was calculated.
%	% calculating percentages of remaining sp
%	With the mean number of extinct and remaining species per step of the extinction sequence, the percentage of remaining species was calculated.
%	% computing means of all webs
%	Finally, the mean over all webs was calculated. Since the extinction scenarios also produced different lengths, the longest extinction sequence per scenario was used as the target length and shorter sequences were padded with zeros to match the target length.
\section{Results}
%There were a number of extinction simulations that were exited prematurely because of improbable interactions. In these cases the algorithm failed to find alternatives to species with improbable interactions and skipped iterations three times (see section \ref{subsec:extc_alg} step \ref{itm:failsafe}). Simulations using abundance based rewiring, showed the highest percentage of aborted runs. Furthermore, especially aTl, atL, and atl networks resulted in frequent termination. Removing species from the lower level generally resulted in lower aborted simulations than removing species from the higher level. See table \ref{tab:abort_perc} in the supplemental materials for an overview. \paragraph{}
% Simulation aborts on lower level
%\begin{table}[H]
%\centering
%\caption{Number of aborted extinction simulations with initial extinction on the lower trophic level. The total number of simulation for each combination of community variable importance and rewiring method was 10000}
%\label{tab:abort_lower}
%\resizebox{\textwidth}{!}{%
%\begin{tabular}{lllllll}
% & org & Atl & aTl & atL & atl & ATL  \\ \hline
%No rewiring & 173  & 139 & 540 & 781 & 999 & 0 \\
%Abundance & 331 & 134 & 1252 & 1391 & 1561 & 0  \\
%Trait & 260 & 185 & 256 & 645 & 770 & 0  \\
%Phylogeny & 12 & 14 & 209 & 281 & 341 & 0  \\
%\begin{tabular}[c]{@{}l@{}}Abundance x \\ Trait\end{tabular} & 295 & 18 & 502 & 497 & 534 & 0  \\
%\begin{tabular}[c]{@{}l@{}}Abundance x \\ Phylogeny\end{tabular} & 69 & 151 & 537 & 775 & 850 & 0 
%\end{tabular}
%}
%\end{table}



% Simulation aborts on higher level
%\begin{table}[]
%\centering
%\caption{Number of aborted extinction simulations with initial extinction on the higher trophic level. The total number of simulation for each combination of community variable importance and rewiring method was 10000}
%\label{tab:abort_higher}
%\resizebox{\textwidth}{!}{%
%\begin{tabular}{lllllll}
% & org & Atl & aTl & atL & atl & ATL \\ \hline
%No rewiring & 245 & 189 & 752 & 1098 & 1367 & 2 \\
%Abundance & 475 & 174 & 1556 & 1864 & 2008 & 1 \\
%Trait & 315 & 253 & 339 & 886 & 1017 & 2 \\
%Phylogeny & 22 & 19 & 268 & 328 & 413 & 1 \\
%\begin{tabular}[c]{@{}l@{}}Abundance x \\ Trait\end{tabular} & 445 & 226 & 666 & 1112 & 1130 & 3 \\
%\begin{tabular}[c]{@{}l@{}}Abundance x \\ Phylogeny\end{tabular} & 118 & 36 & 637 & 694 & 725 & 0
%\end{tabular}%
%}
%\end{table}

%% in percent
%\begin{table}[H]
%\centering
%\caption{Percentage of aborted extinction simulations with initial extinction on the higher trophic level. The total number of simulation for each combination of community variable importance and rewiring method was 10000}
%\label{tab:abort_higher_perc}
%\resizebox{\textwidth}{!}{%
%\begin{tabular}{lllllll}
% & org & Atl & aTl & atL & atl & ATL  \\ \hline
%No rewiring & 2.45 & 1.89 & 7.52 & 10.98 & 13.67 & 0.02  \\
%Abundance & 4.75 & 1.74 & 15.56 & 18.64 & 20.08 & 0.01  \\
%Trait & 3.15 & 2.53 & 3.39 & 8.86 & 10.17 & 0.02  \\
%Phylogeny & 0.22 & 0.19 & 2.68 & 3.28 & 4.13 & 0.01  \\
%\begin{tabular}[c]{@{}l@{}}Abundance x \\ Trait\end{tabular} & 4.45 & 2.26 & 6.66 & 11.12 & 11.30 & 0.03  \\
%\begin{tabular}[c]{@{}l@{}}Abundance x \\ Phylogeny\end{tabular} & 1.18 & 0.36 & 6.37 & 6.94 & 7.25 & 0 
%\end{tabular}%
%}
%\end{table}



Changing the driving force of species interactions during the network simulation, had an effect on the specialisation of species in a network (see figure \ref{fig:h2}). Overall, the interactions in the simulated networks were quite complementary. Using the standard tapnet approach resulted in networks where interactions were largely redundant. On the contrary, the interaction between species in the ATL networks were least conform and $H2'$ values showed the highest variability. In abundance based networks, the interactions between species were the most redundant. Trait and phylogeny based networks were showing a similar degree of generalisation compared to the original networks. Finally, atl networks showed the second highest generalisation.

\begin{figure}[H]
	 \centering
	 \includegraphics[width = .8\textwidth]{/home/leon/Documents/Uni/M.sc/Master Thesis/Networks/models/plot_sink/h2'_all_rev}
	 \captionsetup{width = .8\textwidth}
	 \caption{\textbf{Two dimensional Shannon Entropy of all networks} The boxplots show the H2' values \parencite{Bluethgen2006} for each community variable importance scenario. See section \ref{sec:net_sim} for a detailed explanation of the abbreviations of community variable importances.}
	 \label{fig:h2}
\end{figure}

\subsection{Influence of community variables on network robustness}
% Q1 comparing comvars

To analyse the effect of community variable importance on robustness, networks were compared using no rewiring in the extinction simulation (see plots \ref{fig:extc_cv_norew_lower} and \ref{fig:extc_cv_norew_higher}). There was a clear trend of atl networks being most resistant to extinction cascades, regardless from which trophic level species were removed. Trait and phylogeny based networks showed a relatively low rate of secondary extinctions until a certain percentage of primary extinctions was met, after which secondary extinctions substantially increased. Comparing the species removal scenarios of these three networks showed that under the loss of higher trophic species, secondary extinctions began to impact the network much more early than the removal of lower trophic species. Networks that were simulated using unaltered community variable matrices, were more effected by initial species loss than atl, trait and phylogeny based networks but supported a higher number of species under increasing amount of primary extinctions than abundance based and ATL networks.\\
Extinction curves of abundance based networks had steeper slopes at the beginning and thresholds after which secondary extinctions further increased were less pronounced. Additionally, primary extinctions induced secondary extinctions a lot earlier than for the other networks under both species removal scenarios. The extinction cascades observed in ATL networks were very similar to those of abundance based networks, also showing early increase of secondary extinctions and a less pronounced increase in secondary extinctions with increasing primary extinctions. Trait based and ATL networks showed a high variance over the 10 individual extinction runs for both, removal of lower and higher level species. It should be noted though, that the extinction curves in plots \ref{fig:extc_cv_norew_lower} and \ref{fig:extc_cv_norew_higher} only represent one network, therefore variances might differ for another network.

%old
%This effect was especially present in original, Atl, and ATL networks, when species were removed from the lower trophic level.  Between 99.7 \% (ATL) and 98.1 \% (Atl) of species from the higher trophic level remained after the loss of 75 \% of species from the lower trophic level. However, this trend was less pronounced in the other networks for both species removal scenarios. \par



\begin{figure}[H]
	 \centering
	 \includegraphics[width = \textwidth]{/home/leon/Documents/Uni/M.sc/Master Thesis/Networks/models/plot_sink/extc_sims_cv_norew_lower_rev}
	 \captionsetup{width = \textwidth}
	 \caption{\textbf{Extinction cascades of a single network} The plots shows extinction curves that were simulated for one of the generated networks. Species were removed randomly from the \textbf{lower level} and no rewiring was allowed. Individual extinction simulations are represented by the black lines, while the red line shows their mean. Plot titles indicate the importances of community variables used in the network simulation process. See section \ref{sec:net_sim} for a detailed explanation of the abbreviations of community variable importances.}
	 \label{fig:extc_cv_norew_lower}
\end{figure}


\begin{figure}[H]
	 \centering
	 \includegraphics[width = \textwidth]{/home/leon/Documents/Uni/M.sc/Master Thesis/Networks/models/plot_sink/extc_sims_cv_norew_higher_rev}
	 \captionsetup{width = \textwidth}
	 \caption{\textbf{Extinction cascades of a single network} The plots shows extinction curves that were simulated for one of the generated networks. Species were removed randomly from the \textbf{higher level} and no rewiring was allowed. Individual extinction simulations are represented by the black lines, while the red line shows their mean. Plot titles indicate the importances of community variables used in the network simulation process. See section \ref{sec:net_sim} for a detailed explanation of the abbreviations of community variable importances.}
	 \label{fig:extc_cv_norew_higher}
\end{figure}

There was a clear effect of the importance of community variables during the network simulation regarding robustness (see figure \ref{fig:auc_cv_norew}). The highest robustness was observed in networks were all community variables were considered unimportant. Networks where the importance of all community variables was set to high, showed the lowest robustness. Extinction simulations in original networks resulted in higher robustness than Atl and ATL networks. Simulating species interactions mainly by traits or phylogeny resulted in the third and second highest robustness, respectively. When interactions were determined by abundances, the network robustness was only a little bit higher than what was observed in ATL networks. Generally, networks were less affected by primary extinctions when species were removed from the higher trophic level compared to species removal from the lower level. \par


\begin{figure}[H]
	 \centering
	 \includegraphics[width = .8\textwidth]{/home/leon/Documents/Uni/M.sc/Master Thesis/Networks/models/plot_sink/auc_com_vars_by_norew_rev}
	 \captionsetup{width = .8\textwidth}
	 \caption{\textbf{Network robustness } The boxplots show the robustness values of all extinction simulations without rewiring and random removal of species. See section \ref{sec:net_sim} for a detailed explanation of the abbreviations of community variable importances.}
	 \label{fig:auc_cv_norew}
\end{figure}

% residuals

%old
%The other network simulation scenarios differed substantially. The number of species in the trophic level from which species were removed during extinction simulation, was always the most important effect. While the number of species in the other trophic level was carrying at least some information for the scenario where higher level species were removed, the number of species in the higher trophic level was virtually irrelevant when lower level species were removed. The effects of rewiring method, $H2'$ and their interaction were varying between network simulations scenarios but never explained much variability when species were removed from the lower level. 

\subsection{Network robustness under different rewiring methods}
% Q2 comparing rewiring

The second question focused on the differences between the rewiring methods. Therefore, only networks with the original importances of community variables were used. There was only a small effect of the simulated rewiring methods on network robustness  (see figures \ref{fig:extc_org_rew_lower} and \ref{fig:extc_org_rew_higher}). The only clear difference was apparent when species weren't allowed to compensate for lost interactions through rewiring. This resulted in a higher rate of secondary extinctions under both species removal scenarios.
Regardless of the rewiring method, over 75 \% of species from the higher trophic level remained in the network even after a loss of more than 75 \% of their interaction partners from the lower trophic level. When extinctions were simulated for the higher trophic species and species were allowed to rewire, a loss of 82.5 \% resulted in a decline of less than 25 \% of species in the lower trophic level. Without rewiring the remaining species in the lower level dropped to 68 \%.\\ Similar to the results of the first question, the removal of lower level species led to secondary extinctions earlier than the removal of higher level species. When rewiring methods were combined they only increased network robustness by a small margin compared to the best singular rewiring method (Phylogeny). Unlike for the comparison between network simulation scenarios, the variance between the individual extinction runs per network were similar for each of the rewiring methods. 

\begin{figure}[H]
	 \centering
	 \includegraphics[width = \textwidth]{/home/leon/Documents/Uni/M.sc/Master Thesis/Networks/models/plot_sink/extc_sims_org_rew_lower_rev}
	 \captionsetup{width = \textwidth}
	 \caption{\textbf{Extinction cascades of a single network} The plots shows extinction curves that were simulated for one of the generated networks. Species were removed randomly from the \textbf{lower level} using only original networks. Individual extinction simulations are represented by the black lines, while the red line shows their mean. Plot titles indicate the rewiring method used in the extinction simulation. See section \ref{itm:rew} for a details about the rewiring methods.}
	 \label{fig:extc_org_rew_lower}
\end{figure}


\begin{figure}[H]
	 \centering
	 \includegraphics[width = \textwidth]{/home/leon/Documents/Uni/M.sc/Master Thesis/Networks/models/plot_sink/extc_sims_org_rew_higher_rev}
	 \captionsetup{width = \textwidth}
	 \caption{\textbf{Extinction cascades of a single network} The plots shows extinction curves that were simulated for one of the generated networks. Species were removed randomly from the \textbf{higher level} using only original networks. Individual extinction simulations are represented by the black lines, while the red line shows their mean. Plot titles indicate the rewiring method used in the extinction simulation. See section \ref{subsec:extc_alg} step \ref{itm:rew} for details)}
	 \label{fig:extc_org_rew_higher}
\end{figure}

The difference in mean robustness between the most effective (Abundance x Phylogeny) and least effective rewiring method (no rewiring) was 4.46 when removing lower level species and 4.14 for higher level species (see figure \ref{fig:auc_org_rew}). 

The two scenarios where rewiring methods were combined resulted in the highest mean robustness and were virtually the same. As expected, not allowing species to rewire lost interactions led to the lowest mean robustness. Phylogeny proved to be the most effective singular rewiring method followed by trait based rewiring. Out of all rewiring methods that let species shift interactions to new partners, abundance had the least influence on preventing extinction cascades. 
Underlining the result of the first question, networks were more robust under removal of higher species than of lower species. \par 


\begin{figure}[H]
	 \centering
	 \includegraphics[width = .8\textwidth]{/home/leon/Documents/Uni/M.sc/Master Thesis/Networks/models/plot_sink/auc_org_by_rew_rev}
	 \captionsetup{width = .8\textwidth}
	 \caption{\textbf{Network robustness } The boxplots show the robustness values of all extinction simulations fore each rewiring method and random removal of species. See section \ref{subsec:extc_alg} step \ref{itm:rew} for details) for a detailed explanation of the abbreviations of community variable importances. See section \ref{itm:rew} for a details about the rewiring methods.}
	 \label{fig:auc_org_rew}
\end{figure}

\subsection{Analysis of variance}
% Anova; put after Q1 + Q2 ??
% all
The ANOVA of the whole dataset showed, that changing the importance of community variables to simulate species interactions explained the most variance (see table \ref{tab:anova_full}). The number of species in the lower trophic level explained 5.06 \% of variance. How rewiring partners were chosen explained 3.08 \%. The effects of the other predictors were explaining less than 1 \% of total variance each.

% lower & higher
When an ANOVA was calculated separately for removing species from the lower trophic level and from the higher trophic level, community variable importance and number of species in the lower level were still explaining the most variance (see table \ref{tab:anova_lo_hi}). The explained variance of the rewiring methods was virtually the same as for the whole data set with 3.09 \% and 3.11 \% for the removal of lower and higher species respectively. Furthermore, the unexplained variance did not differ much between species removal scenarios. Again, the effects of species in the higher trophic level, $H2'$ and the interaction between rewiring method and $H2'$ were negligible. \paragraph{}


%For a more detailed insight into what effects were important, an ANOVA was carried out for each network simulation scenario (see table \ref{tab:anova_by_cv} in the supplemental material). As for the full ANOVA, the response was robustness and predictors were, number of lower species, number of higher species, rewiring method, $H2'$ and the interaction between the latter two. The data was subsetted by species removal scenario, which resulted in two ANOVAs per network simulation scenario. \\
%% rew
%When networks were simulated with the original tapnet algorithm, the rewiring method was the most important effect, for either species removal scenario ($R^2_{Original,\: lower} = 23.05\%$ , $R^2_{Original,\: higher} = 13.86\%$). For the other network simulation scenarios the effect of the rewiring method differed greatly. In trait and phylogeny based networks, rewiring only had a marginal effect compared to the other scenarios, irrespective of the level from which species were removed ($R^2_{aTl,\: lower} = 4.30\%$ , $R^2_{aTl,\: higher} = 2.90\%$ and ($R^2_{atL,\: lower} = 5.72\%$ , $R^2_{atL,\: higher} = 4.50\%$)). In Atl networks the effect of rewiring was rather low when lower species were going extinct, but explained almost twice as much variance as all predictors of the latter ANOVA combined when species were removed from the higher level ($R^2_{Atl,\: lower} = 2.77\%$ , $R^2_{Atl,\: higher} = 20.22\%$). A similar, yet opposite relationship was observed for atl networks. There, rewiring alone explained more variance for lower trophic level extinctions than all predictors combined in the higher level exctinction scenario($R^2_{atl,\: lower} = 35.66\%$ , $R^2_{atl,\: higher} = 6.56\%$). Finally, in ATL networks the effect of rewiring were both high compared to aTl and atL networks. Explained variance from rewiring methods was again more than double under the removal of higher species compared to the removal of lower species($R^2_{ATL,\: lower} = 15.28	\%$ \& $R^2_{ATL,\: higher} = 36.93\%$).
%
%%n low
%The number of species in the higher level explained the second most variance for either species removal and network simulation scenario (see table \ref{tab:anova_by_cv}). It explained similar amounts of variance except for four cases. The effect was almost zero in original networks when higher species were removed, and noticeably low for abundance based networks ($R^2_{Original,\: higher} = 0.06\%$ \& $R^2_{Atl,\: higher} = 2.35\%$). Under the extinction of lower level species, the effect was more than double of the maximum of all other network scenarios in atl networks, and again very low in ATl networks($R^2_{atl,\: lower} = 19.75 \%$ , $R^2_{ATL,\: lower} = 0.34\%$).
%
%% n high
%There was a more distinguished trend of variance explained by the number of species in the higher trophic level, as it was always higher under extinction of higher level species, with the exception of original networks. Overall this effect was generally low for all network simulation scenarios. However, there was again an exception, when higher species were removed from ATL networks the effect explained 12.65 \% of variance. 
%
%% h2 and rew x h2
%The other two predictors, $H2'$ as well as the interaction between rewiring method and $H2'$ were never explaining notable amounts of variance. \paragraph{}
%
%In addition to the already presented ANOVAs, the effects of the number of species in their respective trophic level, the influence of community variables on species interactions as well as $H2'$ were used to predict robustness using data subsetted by rewiring methods. For each rewiring scenario an ANOVA was calculated for the two species removal scenarios (see table \ref{tab:anova_by_rew}in the supplemental material). \\
%
%As for the whole dataset, the most variance was explained by the network simulation scenarios. The highest effect was observed under the removal of lower trophic species when abundance and traits were used as the rewiring scenario. ($R^2_{AT,\: lower} = 57.31 \%$). Networks where no rewiring was allowed and also lower species were removed, showed the lowest $R^2$ value with 51.41 \%. The variance explained by changing the influence of community variables to determine species interactions, was not differing consistently between species removal scenarios. Simulations without rewiring and with abundance based rewiring resulted in higher $R^2$ values under the removal of higher species than removal of lower, yet their difference was small ($R^2_{No \: rewiring,\: lower} = 51.41\%$ , $R^2_{No \: rewiring,\: higher} = 53.30\%$ and $R^2_{Abundance,\: lower} = 54.27\%$ , $R^2_{Abundance,\: higher} = 55.13\%$). The other rewiring methods showed the opposite result regarding species removal scenario. However, combining rewiring methods, seemed to enhance the effect of community variables, at least under loss of lower trophic species. \\
%
%The number of species in the lower level was consistently explaining more variance when species from this trophic level were removed rather than species from the higher trophic level. Furthermore, this effect explained the second most variance in all of the ANOVAs discussed in this section. While the number of lower species always carried some information, the number of species in the higher level did not. When removing species from the lower level, the effect was explaining less than 1\% of variance over each of the rewiring scenarios. This changed slightly when species were removed from the higher level, where $R^2$ values ranged between 1.14 \% (Abundance combined with Phylogeny) and 2.19 \% (without rewiring).\\
%
%Similar to the ANOVA compartmentalized by rewiring scenarios, $H2'$ did not provide any significant amount of information on the variance in robustness values under any scenario.

\begin{table}[H]
\centering
\captionsetup{width = \linewidth}
\caption{ANOVA tables. Table a presents the results of an ANOVA for the whole dataset. Table b used subsets of the data depending on which species were removed during the extinction simulations. N\textsubscript{lower} and N\textsubscript{higher} are the number of species in the respective trophic level, Community variables are the species interaction scenarios as described in section \ref{sec:net_sim}, Rewiring methods refer to the rewiring methods explained in section \ref{subsec:extc_alg} step \ref{itm:rew}, $H2'$ is the two dimensional shannon entropy \parencite{Bluethgen2006}, Rewiring method x $H2'$ is the interaction between the latter two, and Trophic level represents the trophic level on which extinction was simulated.}
\begin{subtable}{\linewidth}
\caption{Full data set}
\label{tab:anova_full}
\begin{tabularx}{\linewidth}{@{} X *3{c} @{}}
\toprule
  						& Df 	& Sum of squares	& $R^2$ 		\\ \midrule
Residuals 				& 719835 & 18224087 		& 32.37 \% 	\\ 
N\textsubscript{lower} 	& 1 & 1122299 		& 1.99 \% 	\\
N\textsubscript{higher} 	& 1 & 95541 			& 0.17 \% 	\\
Community variables 		& 5 & 34897751 		& 61.86 \% 	\\
Rewiring method 			& 5 & 1628065 		& 2.88 \% 	\\ 
$H2'$ 					& 1 & 8846			& 0.01 \% 	\\
Rewiring method x $H2'$ 	& 5 & 30600 			& 0.05 \% 	\\ 
Trophic level 			& 1 & 367608 		& 0.65 \%  \\ \bottomrule
\end{tabularx}
\end{subtable}
\medskip


\begin{subtable}{\linewidth}
\caption{Data subsetted by species removal in extinction simulation}
\label{tab:anova_lo_hi}
\begin{tabularx}{\linewidth}{@{} X *6{c} @{}}
\toprule
  & \multicolumn{3}{c}{Lower} & \multicolumn{3}{c}{Higher} \\ \cmidrule(l){2-4} \cmidrule(l){5-7} 
  						& Df 	& Sum of squares	& $R^2$ 		& Df 	& Sum of squares 	& $R^2$  \\ \midrule
Residuals 				& 359921 & 9432633 		& 32.06 \% 	& 352341	& 8731155 			& 32.80 \%\\ 
N\textsubscript{lower} 	& 1 		& 692679 		& 2.35 \% 	& 1 		& 443456 			& 1.66 \%\\
N\textsubscript{higher} 	& 1 		& 3820 			& 0.01 \% 	& 1 		& 140866 			& 0.53 \%\\
Community variables 		& 5 		& 18406451 		& 62.56 \% 	& 5 		& 16521384 			& 62.06 \%\\
Rewiring method 			& 5 		& 863462			& 2.93 \% 	& 5 		& 766266				& 2.87 \%\\ 
$H2'$ 					& 1 		& 8487 			& 0.03 \% 	& 1 		& 1672 				& 0.01 \%\\
Rewiring method x $H2'$ 	& 5 		& 14484 			& 0.05 \% 	& 5 		& 16433				& 0.06 \%\\ \bottomrule
\end{tabularx}
\end{subtable}
\medskip
\end{table}


	\section{Discussion}
% Shortly repeat the Question(s)
To disentangle the effects of rewiring and on the propagation of extinctions in networks, 
% Summarize the main findings
The results of this study showed that the method used to choose a new partner to reallocate interactions after an extinction event only had a small effect on network robustness. However, the mechanisms that were used to determine species interactions during the network simulations exhibited a strong influence on robustness.
% SHORTEN TO SMALL PARAGRAPH !!!!		
The importance of abundance, traits and phylogeny when simulating species interactions, had an effect on the general specialisation of the networks as well as their robustness when no rewiring was allowed (see figures \ref{fig:h2} and \ref{fig:auc_cv_norew}). Furthermore, removing species from the higher level always led to higher network robustness than removing species from the lower level. When species interactions were mainly based on abundance or phylogeny, the interactions in these networks became more unique. This was also the case when abundance, trait and phylogenetic information was used to simulate species interactions but their influence was kept low. Under the assumption that species mainly happen to interact due to matching traits, the simulations showed that the interactions in these networks were rather redundant. This suggests that niches in these networks are overlapping. The highest niche overlap was observed when the probability of an interaction between species was artificially increased for all community variables.\paragraph{}

The propagation of extinctions through networks was lowest when networks were simulated with an amplified influence of all three community variables (see figure \ref{fig:auc_cv_norew}). Assuming that either traits or phylogeny were the main driver of species interactions, led to substantially lower network robustness compared to the networks fitted with the default tapnet approach. Robustness in abundance based networks was only slightly lower than the defualt approach. Networks where the interaction probabilities of all community variables were set to be low, showed the lowest robustness.\\
Furthermore, especially trait, and phylogeny based networks as well as the networks where interaction probabilities of all community variables were lowered, showed a high variance in network robustness. \par

The rewiring method used in the extinction simulation with unaltered networks also had an effect on their robustness (see figure \ref{fig:auc_org_rew}). As expected, not allowing species to reallocate their interactions after losing a partner, facilitated secondary extinctions. When the most abundant species in the network was chosen as a rewiring partner, network robustness was increased. Shifting the interactions of an extinct species to its closest phylogenetic resemblant, showed the highest effect in inhibiting extinction cascades when only one rewiring method was used. Under the assumption that the species with the most similar traits to the extinct species would be chosen as a new interaction partner, the network robustness was higher than the abundance scenario but lower than phylogeny and combined rewiring methods. When species were allowed to compensate lost interactions by shifting them based on a combination of rewiring methods, the robustness was highest. The two combinations of rewiring methods could be regarded equally effective, yet combining abundance and trait was marginally lower than using abundance and phylogeny. \paragraph{}

Changing how species form interactions with one another, by altering the probabilities of abundance, traits, and phylogeny during the network simulation process, explained the most variance in network robustness (see table \ref{tab:anova_full}). How species chose their rewiring partner only explained a very small amount of variance. The only other effect that held some information was the number of species in the lower trophic level. However, both the rewiring method and the number of species in the lower trophic level were lower than community variable importance by a magnitude of over 10. Total amount of variance explained was 62.54 \%.

% H2
Networks where species interactions were mainly determined by abundance, showed the lowest $H2'$ values, suggesting a high niche overlap (see \ref{fig:h2}). This result is in line with the assumption of neutral theory and has been observed in previous studies \parencite{Vazquez2005, Vazquez2007}. \\
Furthermore, simulating species interactions based predominantly on trait-matching, led to substantially higher specialisation compared to abundance based networks. This was also observed for the assumption that species interactions chiefly stem from phylogenetic relatedness. Both of these results underline what is posited by niche theory. Instead of being the product of random encounter, interactions between species are an amalgam of biological factors \parencite{Jordano2003, Rezende2007, Vazquez2009, Olesen2011}. \par

When all community variable matrices were exponentiated with 1.9 they became more homogenous, and in turn did the resulting interaction probabilities (i.e their linear combination). This means interactions in these networks were more or less evenly distributed among all species, which is also the reason for their low specialisation. If, on the other hand, the abundance, trait and phylogeny matrices were exponentiated with 0.1, the interactions that were predicted to be most likely were emphasized. This meant that interactions that were less likely to happen due to one of the community variables, became even more improbable, leading to the higher specialisation observed for these networks. 
\paragraph{}

% Q1
% low H2 should mean high robustness
More redundancy among interactions should, in theory, lead to a higher chance of finding an adequate new partner to compensate lost interactions and thus to higher robustness. However, networks with abundance based interactions were not showing an increased robustness as one would expect by their low $H2'$ values. In fact, the interrelation between $H2'$ and robustness seemed to be the opposite of what was expected (compare \ref{fig:h2} and \ref{fig:auc_cv_norew}). Interestingly, when comparing the robustness of networks in relation to each other they supported the original assumption, except for original and Atl networks.\\ A factor that can lead to lower than expected network robustness is the number of species with only one interaction partner. These singleton species are the most vulnerable to coextinction, since they have to switch their interaction partner successfully in order to survive. This means that if there are a lot of singletons in a network, even though there is a high niche overlap, robustness might still be lower than what would be expected purely from looking at network specialisation. This seemed to be at least partly responsible for the high generalisation in abundance based networks, as they showed the second highest percentage of singeltons after ATL networks (see \ref{fig:ps_by_cv} in the supplemental material). Though, only because there was a higher percentage of singletons in a network that doesn't necessarily translate to an effect on robustness. By visualizing the effect of the percentage of singletons on robustness, it was shown that for abundance based networks, robustness does decrease with increasing precentage of singletons (see figures \ref{fig:r_by_ps_lower} and \ref{fig:r_by_ps_higher} in the supplemental material).


%is most likely explained by the high number of terminated simulation runs in these scenarios (see table \ref{tab:abort_perc}). The high variability in extinction simulations of the individual networks leads to.

% r lower < r higher;confirmed
 This can be interpreted as a greater dependence of higher trophic species on lower trophic species, and was previously reported in other studies \parencite{Bascompte2006, Schleuning2016}.

%The networks simulated with abundance as the main driver of species interactions follow the assumption of neutral theory \parencite{Vazquez2005, Vazquez2007}. It proposes that species are 


%
% generalisation based adaptability of species to determine no of rewired interactions individually
Setting the amount of interactions a species can reallocate during rewiring to a fixed percentage, might bias network robustness, since generalists should more effectively translocate their interactions to other partners than specialists \parencite{Ramos-Jiliberto2012}. Linking the generalisation of a species to its adaptability (see section \ref{subsec:extc_alg}, step \ref{itm:shift}) would be a possibility to remove said bias, but would need experimental data to quantify the effect of generalisation on rewiring success. Baumgartner et al. \parencite{Baumgartner2020} developed an extinction algorithm, that accounted for the importance of an given interaction to a species. This constant was called intrinsic demographic dependence and was used to calculate susceptibility to coextincion in Baumgartner's algortihm. However, this variable could also be used to quantify how effective a species will be reallocating it's interactions to new partners.

VB2019 found that rewiring in pollination networks is largely due to morphological matching. 


% connectance & connectivity vary sig. between biogeographic regions (oelsen&jordano2002) but majority of sampled networks are coming from south america (e.g. brasil), apparently little data for temperate climate (i.e. europe or north america)
%In addition to the mentioned sampling biases, there also seems to be a bias in sampling location. In the case of plant-pollinator networks, South America is clearly overrepresented. For example, half of the available data on plant-pollinator networks on the Interaction Web Database (\url{http://www.ecologia.ib.usp.br/iwdb/index.html}) come from Brazil, Argentina, Chile, Venezuela, and Ecuador. This might have influenced our common understanding of interactions networks since connectance and connectivity were shown to significantly vary between biogeographic regions \parencite{Oelsen2002}.


% see rezende2007 for phylogenetic rewiring !!
Phylogenetically closely related species often show similar traits and the set of species they interact with also tends to overlap \parencite{Rezende2007, Gomez2010}. The difference in robustness for trait and phylogeny based networks should thus be small. This was the case when species were removed from the lower level where the difference in means for trait and phylogeny based networks was the smallest (see fig. \ref{fig:auc_cv_norew}). 
Phylogeny has been identified to play a key role in the formation of interactions in bipartite networks \parencite{Rezende2007}. The authors showed that phylogenetic relatedness was predicting the identity of species' interaction partners in 46.6 \% of networks. They also found, that the number of interactions of phylogenetically related species was similar in 39 \% of networks.

% restrict formation of new interactions during rewiring process !
The extinction algorithm did not restrain species to a maximum number of interaction partners, so that species could freely form new interactions as long as the interaction probability was high enough. Under the assumption that extinction events occur over a short time interval, this could be possible if the observed network is extremely generalised or species could diversify their foraging behaviour almost arbitrarily at any given time. Whilst there is evidence that current species extinction rates are alarmingly high and are bound to increase further \parencite{Ceballos2015}, so that a rapid succession of extinction event might be plausible. It is rather unlikely to observe real-world networks with such a high degree of generalisation or such rapid behavioural changes. However, if extinction events were thought to happen on a larger time scale, for example one extinction per year, the scenario becomes more tenable. \citeauthor{Petanidou2008} found that species that were regarded specialists in one year, tended to be generalists in a following year. By slightly altering the extinction algorithm to include spatio-temporal rewiring processes, such as interaction and species turnover, a consecutive study could easily be implemented. Various studies found, that these two processes are influencing networks, and should thus prove to be a solid foundation\parencite{Poisot2012, Morente-Lopez2018, Brimacombe2021}. Interaction turnover for example, could be simulated by redrawing all interactions after an extinction event and not only those of species that rewired due to losing a partner, as it was done in this study. Furthermore, by drawing from a pool of species and only using subsets during each redraw step of the algorithm, species turnover could be implemented.

Another option would have been to restrict the number of new possible partners for a species. Since temporal variance in interactions was reported in many studies due to a variety of reasons (see for example \cite{Olesen2008, CaraDonna2017, Schwarz2021}), doing so would not be trivial. Rather than restrict the number of partners, Baumgartner et al. limited the number of interactions between species to their original degree \parencite{Baumgartner2020}. 
% Seasonality influences species interactions particularly trhough rewiring. trits linked dto trophic level could be a good indicator of a species contribution to rewiring Brimacombe2021

In VB when multiple attempts with multiple partners are allowed, number of attempts is the number on interactions observed between the species and the lost partner.

Baumgartner restricted NUMBER of interactions between species to original degree of species.
%Further complicating the issue is the fact that, while species and their interactions in a network are changing between years, the structure of the networks is relatively stable \parencite{}.


% petanidou: species are specialists in one year and generalists in another year -> might support that species are very flexible in the number of interaction partners
good approach to simulate extinction cascades over many years, since redrawing networks after an extinction can be a proxy for interannual variability in species interactions and species can be specialists in one year and generalists in another year (reason for not restricting number of new interaction partners, see petanidou2008). would additionally need some form of species turnover, e.g. by randomly drawing from a pool of species each extinction iteration.

% construct validity !! can results be translated from simulations into reality ? does it make sense to assume these connections/interrelationships/results in real networks ?
Finally, it should be considered how well the results of this analysis can be translated to real-world networks. While there is ample support of species interactions being driven by neutral and niche processes \parencite{Jordano2003, Rezende2007, Vazquez2007, Bluethgen2008}, it is unlikely that interactions are solely formed by either, abundance, traits or phylogeny as was simulated here. However, since the main goal of this study was to investigate underlying
connections of species interaction drivers and rewiring method, and their effects on network robustness, networks don't have to resemble real-world communities.

% only "quantitative" change in com_vars, could be interesting to change "qualitatively" i.e. abundance distribution (log-normal, normal, etc.), phylogeny (branching-early / branching late)...
\section{Conclusion}
Thus, while being a powerful tool, simulating extinctions with artificial networks should be done with great caution. Users should meticulously think about what the drivers of interactions in the particular networks might be and how rewiring might be restricted. Nonetheless, this approach delivered information regarding the influence of abundance, traits, and phylogeny on the formation of species interactions and their role in finding new interaction partners, by untangling their effect on network robustness.
\section{Acknowledgments}
\newpage
\section*{Declaration of Originality}
I, Leon Thoma hereby declare that the presented thesis is my own work and that I have not sought or used inadmissible help of third parties to produce this work. Furthermore, I have clearly referenced all sources used in the work and used inverted commas for all text directly or indirectly quoted from a source.\paragraph{}
This work has not been submitted to another examination institution - neither in Germany nor outside Germany - neither in the same nor in a similar way and has not been published.\paragraph{}

Freiburg,\paragraph{}

\rule{5cm}{.4pt}\par
Leon Thoma
\newpage
\section{Supplemental material}

\begin{figure}[H]
	 \centering
	 \includegraphics[width = .8\textwidth]{/home/leon/Documents/Uni/M.sc/Master Thesis/Networks/models/plot_sink/ps_by_com_vars_rev}
	 	 \captionsetup{width = .8\textwidth}
	 \caption{\textbf{Percentage of singletons by community variables} The boxplots show the robustness values of all extinction simulations fore each rewiring method and random removal of species. See section \ref{subsec:extc_alg} step \ref{itm:rew} for details) for a detailed explanation of the abbreviations of community variable importances. See section \ref{itm:rew} for a details about the rewiring methods.}
	 \label{fig:ps_by_cv}
\end{figure}

\begin{figure}[H]
	 \centering
	 \includegraphics[width = .8\textwidth]{/home/leon/Documents/Uni/M.sc/Master Thesis/Networks/models/plot_sink/r_by_ps_lower_rev}
	 \captionsetup{width = .8\textwidth}
	 \caption{\textbf{Robustness by mean percentage of singletons} lower}
	 \label{fig:r_by_ps_lower}
\end{figure}

\begin{figure}[H]
	 \centering
	 \includegraphics[width = .8\textwidth]{/home/leon/Documents/Uni/M.sc/Master Thesis/Networks/models/plot_sink/r_by_ps_higher_rev}
	 	 \captionsetup{width = .8\textwidth}
	 \caption{\textbf{Robustness by mean percentage of singletons} higher}
	 \label{fig:r_by_ps_higher}
\end{figure}


%%%%%%%%%%%%%%%%%%%%%

%\captionsetup{width = \linewidth}
%\caption{Percentage of aborted extinction simulations with initial extinction on the lower trophic level. The total number of simulation for each combination of community variable importance and rewiring method was 10000}
%\label{tab:abort_perc}
%\begin{tabularx}{\linewidth}{Xllllllllllll}
%\toprule
%  & \multicolumn{6}{c}{Lower} & \multicolumn{6}{c}{Higher} \\ \cmidrule(l){2-7} \cmidrule(l){8-13}
% & org & Atl & aTl & atL & atl & ATL & org & Atl & aTl & atL & atl & ATL \\ \midrule
%No rewiring & 1.73 & 1.39 & 5.40 & 7.81 & 9.99 & 0 & 2.45 & 1.89 & 7.52 & 10.98 & 13.67 & 0.02  \\
%Abundance & 3.31 & 1.34 & 12.51 & 13.91 & 15.61 & 0 & 4.75 & 1.74 & 15.56 & 18.64 & 20.08 & 0.01 \\
%Trait & 2.60 & 1.85 & 2.56 & 6.45 & 7.70 & 0 & 3.15 & 2.53 & 3.39 & 8.86 & 10.17 & 0.02 \\
%Phylogeny & 0.12 & 0.14 & 2.09 & 2.81 & 3.41 & 0 & 0.22 & 0.19 & 2.68 & 3.28 & 4.13 & 0.01 \\
%Abundance x Trait & 2.95 & 0.18 & 5.02 & 4.97 & 5.34 & 0 & 4.45 & 2.26 & 6.66 & 11.12 & 11.30 & 0.03 \\
%Abundance x Phylogeny & 0.69 & 1.51 & 5.37 & 7.75 & 8.50 & 0 & 1.18 & 0.36 & 6.37 & 6.94 & 7.25 & 0 \\ \bottomrule
%\end{tabularx}%
%
%\vspace{0.8cm}
%%($R^2_{N_higher} = 0.67 %$, $R^2_{Trophic_level} = 0.55 %$, $R^2_{Rewiring} = 0.34 %$, $R^2_{H2'} = 0.33 %$, $R^2_{Rewiring x H2'} = 0.05%$)
%
%%%%%%%%%%%%%%%%%%%%%%


%%%%%%%%%%%%%%%%%%%%%
\begin{table}[H]
\label{tab:anova_by_cv}
\caption{ANOVA tables by network simulation scenario. The first line of the table states the subset of data used. 'Lower' refers to simulations where species were removed from the lower trophic level and 'Higher' to simulations were species were removed from the higher trophic level. N\textsubscript{lower} and N\textsubscript{higher} are the number of species in the respective trophic level, Rewiring methods refer to the rewiring methods explained in section \ref{subsec:extc_alg} step \ref{itm:rew}, $H2'$ is the two dimensional shannon entropy \parencite{Bluethgen2006}, and Rewiring method x $H2'$ is the interaction between rewiring methods and $H2'$}
    \begin{subtable}{\linewidth}
    \caption{Original}
    \centering
\begin{tabularx}{\linewidth}{@{} X *6{c} @{}}
\toprule
  & \multicolumn{3}{c}{Lower} & \multicolumn{3}{c}{Higher} \\ \cmidrule(l){2-4} \cmidrule(l){5-7}
  						& Df		& Sum of squares	& $R^2$	& Df 	& Sum of squares	& $R^2$ \\ \midrule
Residuals 				& 59836 & 769722			& 83.68	& 59876 & 736718 		& 85.41   \\
N\textsubscript{lower} 	& 1 		& 5197 			& 0.56 	& 1		& 666 			& 0.08  \\
N\textsubscript{higher} 	& 1 		& 747 			& 0.08	& 1 		& 407 			& 0.05  \\
Rewiring method 			& 5 		& 140112 		& 15.23 	& 5 		& 119821			& 13.89  \\
$H2'$ 					& 1 		& 3292 			& 0.36 	& 1 		& 4352			& 0.50 \\
Rewiring method x $H2'$ 	& 5 		& 724 			& 0.08 	& 5 		& 615			& 0.07  \\ \bottomrule
\end{tabularx}
\end{subtable}

%%%%%%%%%%%
\medskip
\begin{subtable}{\linewidth}
\caption{Atl}
\centering
\begin{tabularx}{\linewidth}{@{} X *6{c} @{}}
\toprule
  & \multicolumn{3}{c}{Lower} & \multicolumn{3}{c}{Higher} \\ \cmidrule(l){2-4} \cmidrule(l){5-7}
  						& Df		& Sum of squares	& $R^2$	& Df 	& Sum of squares	& $R^2$ \\ \midrule
Residuals 				& 58096 & 4196776		& 85.11	& 59806 & 3776235 		& 85.99   \\
N\textsubscript{lower} 	& 1 		& 457068 		& 9.27 	& 1		& 407442		& 9.28  \\
N\textsubscript{higher} 	& 1 		& 166235 			& 3.37	& 1 		& 94172 			& 2.14  \\
Rewiring method 			& 5 		& 108878			& 2.21 	& 5 		& 104919			& 2.39  \\
$H2'$ 					& 1 		& 1790 			& 0.04 	& 1 		& 7033			& 0.16 \\
Rewiring method x $H2'$ 	& 5 		& 536 			& 0.01 	& 5 		& 1629			& 0.04  \\ \bottomrule
\end{tabularx}
\end{subtable}

%%%%%%%%%%%
\medskip
\begin{subtable}{\linewidth}
\caption{aTl}
\centering
\begin{tabularx}{\linewidth}{@{} X *6{c} @{}}
\toprule
  & \multicolumn{3}{c}{Lower} & \multicolumn{3}{c}{Higher} \\ \cmidrule(l){2-4} \cmidrule(l){5-7}
  						& Df		& Sum of squares	& $R^2$	& Df 	& Sum of squares	& $R^2$ \\ \midrule
Residuals 				& 58656 & 771103		& 71.01	& 56966 & 747758		& 73.44   \\
N\textsubscript{lower} 	& 1 		& 105453 		& 9.71 	& 1		& 59791			& 5.87  \\
N\textsubscript{higher} 	& 1 		& 5844 			& 0.54	& 1 		& 37232		& 3.66  \\
Rewiring method 			& 5 		& 185703 		& 17.10 	& 5 		& 153172		& 15.04  \\
$H2'$ 					& 1 		& 12092 			& 1.11 	& 1 		& 17451			& 1.71 \\
Rewiring method x $H2'$ 	& 5 		& 5692 			& 0.52 	& 5 		& 2780			& 0.27  \\ \bottomrule
\end{tabularx}

\end{subtable}
\end{table}

%%%%%%%%%%%

\begin{table}
\ContinuedFloat
\caption{continued}
    \begin{subtable}{\linewidth}
    \caption{atL}
    \centering
\begin{tabularx}{\linewidth}{@{} X *6{c} @{}}
\toprule
  & \multicolumn{3}{c}{Lower} & \multicolumn{3}{c}{Higher} \\ \cmidrule(l){2-4} \cmidrule(l){5-7}
  						& Df		& Sum of squares	& $R^2$	& Df 	& Sum of squares	& $R^2$ \\ \midrule
Residuals 				& 58746 & 264573		& 46.60	& 57786 & 193817 		& 46.35   \\
N\textsubscript{lower} 	& 1 		& 139941 		& 24.65 	& 1		& 27694 		& 6.62  \\
N\textsubscript{higher} 	& 1 		& 1247 			& 0.21	& 1 		& 59662			& 14.27  \\
Rewiring method 			& 5 		& 156858			& 27.63 	& 5 		& 131297			& 31.40  \\
$H2'$ 					& 1 		& 3570 			& 0.63 	& 1 		& 5103			& 1.22 \\
Rewiring method x $H2'$ 	& 5 		& 1528 			& 0.27 	& 5 		& 553			& 0.13  \\ \bottomrule
\end{tabularx}
\end{subtable}

%%%%%%%%%%%
\medskip
    \begin{subtable}{\linewidth}
    \caption{atl}
    \centering
\begin{tabularx}{\linewidth}{@{} X *6{c} @{}}
\toprule
  & \multicolumn{3}{c}{Lower} & \multicolumn{3}{c}{Higher} \\ \cmidrule(l){2-4} \cmidrule(l){5-7}
  						& Df		& Sum of squares	& $R^2$	& Df 	& Sum of squares	& $R^2$ \\ \midrule
Residuals 				& 59986 & 148797			& 43.05	& 57856 & 105048		& 44.02   \\
N\textsubscript{lower} 	& 1 		& 84779 			& 24.53 	& 1		& 14656			& 6.14  \\
N\textsubscript{higher} 	& 1 		& 492 			& 0.14	& 1 		& 31701 		& 13.28  \\
Rewiring method 			& 5 		& 108911 		& 31.51 	& 5 		& 85565			& 35.85  \\
$H2'$ 					& 1 		& 1661 			& 0.48 	& 1 		& 1425			& 0.60 \\
Rewiring method x $H2'$ 	& 5 		& 967 			& 0.28 	& 5 		& 240			&  0.10  \\ \bottomrule
\end{tabularx}
\end{subtable}

%%%%%%%%%%%
\medskip
    \begin{subtable}{\linewidth}
    \caption{ATL}
    \centering
\begin{tabularx}{\linewidth}{@{} X *6{c} @{}}
\toprule
  & \multicolumn{3}{c}{Lower} & \multicolumn{3}{c}{Higher} \\ \cmidrule(l){2-4} \cmidrule(l){5-7}
  						& Df		& Sum of squares	& $R^2$	& Df 	& Sum of squares	& $R^2$ \\ \midrule
Residuals 				& 59896 & 2970402		& 82.91	& 59986 & 2854801			& 82.95   \\
N\textsubscript{lower} 	& 1 		& 339818 		& 9.48 	& 1		& 244337 			& 7.10  \\
N\textsubscript{higher} 	& 1 		& 57389 			& 1.60	& 1 		& 133933 			& 3.89  \\
Rewiring method 			& 5 		& 195005 		& 5.44 	& 5 		& 202483			& 5.88  \\
$H2'$ 					& 1 		& 19502 			& 0.54 	& 1 		& 5088			& 0.15 \\
Rewiring method x $H2'$ 	& 5 		& 578 			& 0.02 	& 5 		& 1103			& 0.03  \\ \bottomrule
\end{tabularx}

\end{subtable}

\end{table}


%%%%%%%%%%%%%%%%%%%%%
\begin{table}[H]
\label{tab:anova_by_rew}
\caption{ANOVA tables by rewiring method. The first line of the table states the subset of data used. 'Lower' refers to simulations where species were removed from the lower trophic level and 'Higher' to simulations were species were removed from the higher trophic level. N\textsubscript{lower} and N\textsubscript{higher} are the number of species in the respective trophic level, Community variables are the species interaction scenarios as described in section \ref{sec:net_sim} and $H2'$ is the two dimensional shannon entropy \parencite{Bluethgen2006}}
    \begin{subtable}{\linewidth}
    \caption{No rewiring}
    \centering
\begin{tabularx}{\linewidth}{@{} X *6{c} @{}}
\toprule
      & \multicolumn{3}{c}{Lower} & \multicolumn{3}{c}{Higher} \\ \cmidrule(l){2-4} \cmidrule(l){5-7}
  						& Df		& Sum of squares	& $R^2$	& Df 	& Sum of squares	& $R^2$ \\ \midrule
Residuals 				& 57691 & 2132294		& 42.49	& 56481 & 1804118 		& 41.03   \\
N\textsubscript{lower} 	& 1 		& 272181			&  5.42	& 1		& 136953			& 3.11  \\
N\textsubscript{higher} 	& 1 		& 6343			& 0.13	& 1 		& 96257 			& 2.19  \\
Community variables 		& 5 		& 2580408 		& 51.41	& 5 		& 2343401			& 53.30  \\
$H2'$ 					& 1 		& 27511			& 0.55 	& 1 		& 16001		& 0.36 \\ \bottomrule
\end{tabularx}
 \end{subtable}

%%%%%%%%%%% 
 \medskip
\begin{subtable}{\linewidth}
\caption{Abundance}
    \centering
\begin{tabularx}{\linewidth}{@{} X *6{c} @{}}
\toprule
  & \multicolumn{3}{c}{Lower} & \multicolumn{3}{c}{Higher} \\ \cmidrule(l){2-4} \cmidrule(l){5-7}
  						& Df		& Sum of squares	& $R^2$	& Df 	& Sum of squares	& $R^2$ \\ \midrule
Residuals 				& 58331 & 2090219		& 39.69	& 57111 & 1753889 		& 39.33   \\
N\textsubscript{lower} 	& 1 		& 295945 		& 5.62 	& 1		& 159327 			& 3.57  \\
N\textsubscript{higher} 	& 1 		& 10068 			& 0.19	& 1 		& 78394 			& 1.76  \\
Community variables		& 5 		& 2858010 		& 54.27 	& 5 		& 2458054			& 55.13  \\
$H2'$ 					& 1 		& 11741 			& 0.22 	& 1 		& 9261			& 0.21 \\ \bottomrule
\end{tabularx}
\end{subtable}

%%%%%%%%%%%
\medskip
    \begin{subtable}{\linewidth}
    \caption{Trait}
    \centering
\begin{tabularx}{\linewidth}{@{} X *6{c} @{}}
\toprule
  & \multicolumn{3}{c}{Lower} & \multicolumn{3}{c}{Higher} \\ \cmidrule(l){2-4} \cmidrule(l){5-7}
  						& Df		& Sum of squares	& $R^2$	& Df 	& Sum of squares	& $R^2$ \\ \midrule
Residuals 				& 59581 & 1899875		& 39.00	& 59431 & 1686462 		& 38.92   \\
N\textsubscript{lower} 	& 1 		& 300785			& 6.17 	& 1		& 227370			& 5.25  \\
N\textsubscript{higher} 	& 1 		& 22390				& 0.46	& 1 		& 68723			& 1.59  \\
Community variables 		& 5 		& 2630660 		& 54.00 	& 5 		& 2338592		& 53.98  \\
$H2'$ 					& 1 		& 18092			&  0.37	& 1 		& 11407			& 0.26 \\ \bottomrule
\end{tabularx}
\end{subtable}
\end{table}

%%%%%%%%%%%

\begin{table}
\ContinuedFloat
%\caption{continued}
    \begin{subtable}{\linewidth}
    \caption{Phylogeny}
    \centering
\begin{tabularx}{\linewidth}{@{} X *6{c} @{}}
\toprule
  & \multicolumn{3}{c}{Lower} & \multicolumn{3}{c}{Higher} \\ \cmidrule(l){2-4} \cmidrule(l){5-7}
  						& Df		& Sum of squares	& $R^2$	& Df 	& Sum of squares	& $R^2$ \\ \midrule
Residuals 				& 59881 & 1824705		& 37.89 	& 59751 & 1674102 		& 38.23   \\
N\textsubscript{lower} 	& 1 		& 280635			& 5.83 	& 1		& 234028 			& 5.34  \\
N\textsubscript{higher} 	& 1 		& 28343				& 0.59	& 1 		& 64080 			& 1.46  \\
Community variables 		& 5 		& 2667901 		& 55.40 	& 5 		& 2399170			& 54.79  \\
$H2'$ 					& 1 		& 13622 			& 0.28 	& 1 		& 7502			& 0.17 \\ \bottomrule
\end{tabularx}
\end{subtable}

%%%%%%%%%%%
\medskip
    \begin{subtable}{\linewidth}
    \caption{Abundance x Trait}
    \centering
\begin{tabularx}{\linewidth}{@{} X *6{c} @{}}
\toprule
  & \multicolumn{3}{c}{Lower} & \multicolumn{3}{c}{Higher} \\ \cmidrule(l){2-4} \cmidrule(l){5-7}
  						& Df		& Sum of squares	& $R^2$	& Df 	& Sum of squares	& $R^2$ \\ \midrule
Residuals 				& 59811 & 1668163		& 35.67	& 59721 & 1554133 		& 37.15   \\
N\textsubscript{lower} 	& 1 		& 291201 		& 6.23 	& 1		& 250412			& 5.98  \\
N\textsubscript{higher} 	& 1 		& 29877			& 0.64	& 1 		& 51944 			& 1.24  \\
Community variables 		& 5 		& 2679731 		& 57.31 	& 5 		& 2323759			& 55.55  \\
$H2'$ 					& 1 		& 7190			& 0.15 	& 1 		& 3140	& 0.07 \\ \bottomrule
\end{tabularx}
\end{subtable}

%%%%%%%%%%%
\medskip
    \begin{subtable}{\linewidth}
    \caption{Abundance x Phylogeny}
    \centering
\begin{tabularx}{\linewidth}{@{} X *6{c} @{}}
\toprule
  & \multicolumn{3}{c}{Lower} & \multicolumn{3}{c}{Higher} \\ \cmidrule(l){2-4} \cmidrule(l){5-7}
  						& Df		& Sum of squares	& $R^2$	& Df 	& Sum of squares	& $R^2$ \\ \midrule
Residuals 				& 59951 & 1719157		& 36.03	& 59811 & 1560658			& 37.29   \\
N\textsubscript{lower} 	& 1 		& 285053			& 5.97 	& 1		& 238023 			& 5.69  \\
N\textsubscript{higher} 	& 1 		& 33508			& 0.70	& 1 		& 47900 			& 1.14  \\
Community variables 		& 5 		& 2726262 		& 57.14 	& 5 		& 2336967			& 55.83  \\
$H2'$ 					& 1 		& 7043			& 0.14 	& 1 		& 2083			& 0.05  \\ \bottomrule
\end{tabularx}
\end{subtable}
\end{table}



%%%%%%%%%%%%%%%%%%%%%
\newpage
\begin{multicols}{2}[\printbibheading]
\printbibliography[heading=none]
\end{multicols}
\end{document}
