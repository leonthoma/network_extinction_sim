\documentclass[12pt,a4paper]{article}
\usepackage[utf8]{inputenc}
\usepackage{csquotes}
\usepackage[english]{babel}
\usepackage[T1]{fontenc}
\usepackage{graphicx}
\usepackage{float}
\author{Leon Thoma}
\title{Master thesis in environmental sciences}
\usepackage[backend=biber, style=authoryear, sorting=none, giveninits=true]{biblatex}%  !!!In bib only state the first char of the first name!!!
\AtBeginBibliography{\footnotesize}
\usepackage{url}
\usepackage[font={footnotesize}, labelfont=bf]{caption}
\usepackage[table,xcdraw]{xcolor}
\usepackage{caption}
\usepackage{amssymb}
\usepackage{amsmath}
\addbibresource{extc_net.bib}
\defbibheading{none}{}
\usepackage[colorlinks]{hyperref}
\hypersetup{
    colorlinks,
    linkcolor={black!50!black},
    citecolor={black!50!black},
    urlcolor={blue!80!black}
}
\usepackage{multicol}
\begin{document}
\begin{titlepage}
	\centering
	\includegraphics[width=0.25\textwidth]{logo-grundversion-01}\par\vspace{1cm}
	{\scshape\large Albert-Ludwigs-Universität Freiburg\par}
	\vspace{1.25cm}
	{\scshape\large Thesis for the attainment of the degree:\\ Master of science\par}
	\vspace{.75cm}
	{\scshape\large Faculty for Environment and Natural Resources\par}
	\vspace{.75cm}
	{\scshape\large Chair of Biometry and Ecosystem Analysis\par}
	\vspace{.75cm}
	{\Large\itshape Rewiring networks under species loss: a simulation
study
\par}
	\vspace{.75cm}
	{\Large\itshape Leon Thoma\par}
	\vspace{.25cm}
	{\scshape\normalsize E-mail: leonthoma@gmx.de\par}
	\vspace{.15cm}
	{\scshape\normalsize Matriculation number: 3733253\par}
	\vspace{.75cm}
	\large Examiner:\par
	\large Prof. Dr. Carsten  \textsc{Dormann} \par
	\vspace{.5cm}
	\large Co-Examiner:\par
	\large Dr. Anne-Christine \textsc{Mupepele}
	\vfill

	{\large \today\par}
\end{titlepage}
	\tableofcontents
	\newpage	
	\section{Abstract} 

\newpage%
%
% RECHECKING:
% GO THROUGH ALL CITATIONS IN JABREF FOR CORRECT SPECIAL CHARACTERS AND GENERAL CHECK
%
\section{Introduction}
Contrary to a main question in the field of invasion ecology, how novel species influence interactions among species, stands the question of how species leaving an ecosystem affects the remaining species. 

% species leaving network can have influence on connection between sp over different biomes/ ecosystems (terrestrial/aquatic; knight 2005 in colwell 2012)
% coextinction is difficult to document (colwell 2012):
	% determining host specifity -> interaction partners might become extinct before their hosts (Moir 2010, 2012b, Powell 2011)
	% if spcies are reintroduced to wild, interactions partners might be affected by treatment used to bring up reintroduced species (e.g. biocides)
	% 

%
%The introduction of species was studied in a wide variety of ecosystems (references here) and across many species (references here), studying the effects of extinctions, however is much harder. This is due to the fact that  
% Intro to topic:
	% network ecology
Network ecology has helped researchers to develop a more detailed understanding ecological communities. Starting from analysing functions and structures within networks and identifying patterns \parencite{Thebault2010} to finding the underlying processes that lead to the observerd patterns. Especially the role of individual species and their interactions was elucidated \parencite{Waser1996, Lau2017}. 
		% lau2017
		% poisot2016; 10.1111/1365-2435.12799
		% vazquez2022
	% extinction simulation
	% Different approaches (Memmot, Kaiser-Bunbury, Vieira, Baumgartner, Vizentin-Bugoni)
The following 3 models mark cornerstones in advances of extinction cascade modelling. In 2004 \citeauthor{Memmott2004} simulated secondary extinction for plant-pollinator networks using three scenarios for initial extinctions of pollinators (but not plants). Random extinction (Null model), sequential removal of species starting with the most linked, and sequential removal starting with the least linked species. A species was considered to be coextinct if it lost all its interactions. However, the approach disregarded a few key points, such as the quality of a given interaction (i.e. pollination success and frequency of visits), the fact that plants can reproduce even without pollinators e.g. by selfing and arguably most important that species don't compensate for lost interaction partners (either by creating new interactions or enhancing existing ones). Nevertheless, the research of \citeauthor{Dunne2002} and \citeauthor{Memmott2004} provided helpful insights into network structure showing that species in less connected networks are more prone to suffer from coextinction. \par

A first quantitative approach to extinction modelling in mutualistic networks considering behavioural shifts (i.e. interaction rewiring) was done by \citeauthor{Kaiser-Bunbury2010} in 2010 \parencite{Kaiser-Bunbury2010}. They based rewiring on interactions between species that were observed at one point of time when the site was sampled. For example, if a species lost all it's interaction partners it would remain in the model with it's full interaction strength if a partner was present with which they observed an interaction in the network at an earlier or later point in time. Their results indicated that networks are more robust to extinction when rewiring is considered since it's 'artificially' increasing connectance. \par

In 2019 \citeauthor{Vizentin-Bugoni2019} extended the approach of \citeauthor{Memmott2004} and \citeauthor{Dunne2002} \parencite{Memmott2004, Dunne2002} for (co)extinction simulation.
% by enhancing the functions provided in the bipartite R-package \parencite{dormann2008}. 

%% Rephrase !!!
Rewiring was implemented as a two step process. First a potential partner is chosen based either on (1) variables known to be important for species interactions in the studied network (e.g. abundance) or (2) at random. Second, the model computes a rewiring probability for the chosen partner, accounting for species abundance, phenological overlap, morphological traits and combinations thereof. Finally, a binomial trail will determine if rewiring is successful or not. Furthermore, the model allows its users to choose the number of rewiring attempts \footnote{
One try with a single partner, 
multiple tries with a single partner, 
one try with each partner, 
multiple tries with multiple partners
}. Following the assumption of \citeauthor{Memmott2004} and \citeauthor{Dunne2002}, a species was considered coextinct if it lost all it's interaction partners.
\par

	% general results of extc sim w/ rewiring
		% robustness increases w/ rewiring: dunne2002, eklöf2006, thebault & fontaine2010, baumgartner2020
The results of research on extinctions and their propagation in ecological networks have shown, that loss of generalists causes a higher rate of coextinction than loss of specialists (\cite{Memmott2004, Kaiser-Bunbury2010, Traveset2017, Bastazini2018, Vizentin-Bugoni2019, Biella2020}; but see \cite{Dunne2002}).
Another result was a substantially lower robustness in mutualistic bipartite networks, when species were removed from the higher trophic level compared to when lower trophic level species were removed \parencite{Schleuning2016, Traveset2017}. This was the case irrespective of how the species that was to be removed was chosen, i.e. random, abundance based, etc.(RECHECK!!!) This is most likely due to the fact, that species' roles in networks are highly asymmetric, meaning species belonging to the higher trophic level are very dependent on their interaction partners from the lower level, but not vice versa \parencite{Bascompte2006}. 
\citeauthor{Dunne2002, Ekloef2006} and, \citeauthor{Thebault2010} found a positive correlation of network robustness and connectance \parencite{Dunne2002, Ekloef2006, Thebault2010}. However, \citeauthor{Vieira2015} showed that the opposite was the case \parencite{Vieira2015}. They argued, that the assumption, that coextinction only occurs when a species loses all its interaction partners, which the previous studies made, led to their finding. Recently, \citeauthor{Baumgartner2020}, challenged the reasoning of \citeauthor{Vieira2015} by showing a positive correlation between network robustness and connectance using an extinction algorithm that accounts for the interaction strengths (i.e. dependence), intrinsic demographic dependence, and dissimilarity of species to estimate coextinction susceptibility \parencite{Baumgartner2020}. 

\parencite{Kaiser-Bunbury2010, Schleuning2016, Timoteo2016, Costa2018} higher robustness due to rewiring. % also (Ramos‐Jiliberto et al.,
%2012; Nuwagaba et al., 2017) from baumgartner2020; compare with blüthgen2010 and vieira2015


% Explaining problem/ gap of knowledge
Despite the insights previous studies provided, it is still unclear if rewiring is influenced by community structures, which then affects the number of secondary extinctions in a network. To provide further insight into the mechanistics behind rewiring, this study will focus on modelling extinction cascades using simulated networks with controlled community structures. \\

% For example, using the same rewiring criteria, does robustness change when the main driver of interactions in the network is changed ? Its still unknown if e.g. the abundance of species in a network changes the propagation of extinctions. 
	% rewiring methods
	not clear how different rewiring mechanisms affect extinction propagation. especially in bugoni et al combining rewiring mechanisms led to lower effect size than single rewiring mechanisms

% Explain the approach
The analysis of this study was a two step process. First, networks were simulated and then used in the extinction algorithm to assess network robustnesses (see fig. \ref{fig:extc_alg}).  
Network simulations were based on the approach presented by \citeauthor{Benadi} \parencite{Benadi}. The 'observed' interactions in a network are simulated by combining abundance, trait, and phylogenetic information. The influence of the three variables on the formation of interactions between species was altered by using either one, none or all of them as the main driver.

	% rewiring methods
	finding out how different rewiring methods influence extinction cascades in networks. therefore select one com var to be main driver of interactions in network and compare the rew methods. 
	
	simulating networks has the advantage of equal "sampling effort" and is therefore not affected by sampling biases !!
% Research questions
Does network robustness change for different rewiring methods when the driving force of interactions in networks is altered ?

I hypothesize that matching community structure and rewiring method (e.g. Atl and abundance based rewiring) will lead to higher network robustness compared to networks where the community structure and rewiring method are not overlapping. However, since phylogenetically closely related species often show similar traits, the set of species they interact with also tends to overlap \parencite{Gomez2010, Rezende2007}. The difference in robustness for trait and phylogeny based networks should thus be small when using trait or phylogeny based rewiring. 
\newpage
	\section{Material \& Methods}
	The simulations in this study were executed in R \parencite{Rcore} (version 3.6.3) based on the functions of the tapnet\footnote{The paper is yet to be published, but the package and it's vignette are available via CRAN and GitHub; https://cran.r-project.org/web/packages/tapnet/index.html, https://github.com/biometry/tapnet} \parencite{Benadi} and bipartite \parencite{Dormann2008} packages as well as the approach of the extinction algorithm developed by \citeauthor{Vizentin-Bugoni2019} \parencite{Vizentin-Bugoni2019}. Some of the functions provided by these sources were modified or augmented to match the specific needs of the presented approach. The source code of this analysis is available in the supplementary materials. \par


% describe general approach:
	% initial network sim
	% sim webs w/ com_var
	% choose rewiring method & extinction method
	% run extc models

	\subsection{Network simulation} \label{sec:net_sim}
	To simulate the networks, each of the three community variables were provided as an independent two-dimensional matrix defined by the species of the two trophic levels. First, interaction frequencies were derived from the linear combination of the three community variable matrices, which are all scaled to sum of one before and after their multiplication. Interaction frequencies were then used as probabilities in a multinomial draw to calculate the desired number of 'observed' interactions (see also fig. 1 in \cite{Benadi} for a conceptual overview). \par

	
	% first simulate initial network with set no of sp in lower and upper trophic level, no of traits w/ and w/o pems -> done with simulate_tapnet_aug
	The number of species in the trophic levels of the initial network was 20 and 30 for the lower and higher level, respectively. Total number of observations was set to 1111. The number of simulated traits was set to 2 for phylogenetically correlated and uncorrelated traits each.
	% caradonna: nobs: 30000, ntraits: 2, nlow: , nhigh:
	% kaiser-bunbury: nobs: 1278, ntraits: NA, nlow: , nhigh:
	% costa: nobs: 	3974, ntraits: NA, nlow: , nhigh:
	% bugoni: nobs: na, ntraits; 2, nsims:1000, nlow: , nhigh
	% schleuning:   see supp mat.
	% memmot (10.1046/j.1461-0248.1999.00087.x): nobs: 2183, nlow: , nhigh: 
	% vazquez/simberloff (10.1046/j.1461-0248.2003.00534.x): nobs: 5285, nlow: , nhigh:
	

	% using simulated abundances, traits and other params (paramlist) simulate omat for all com importance scenarios
	Using the abundance \& trait matrices and the phylogenetic eigenmatrices from the initial simulation, five networks were simulated with 1000 simulations each. Additionally, either one, all or none of the three community variables (abundance (a), traits (t), and phylogeny/latent traits (l)) used to simulate interactions were selected as a main driver. This was achieved by exponentiating the corresponding matrix with either 0 or 1.9, resulting in high or low importance respectively. If none of the variables were regarded as especially important, matrices were left unchanged (i.e. the exponent was 1). The five networks will be referred to as Atl, aTl, atL, ATL and, atl with capital letter(s) indicating a main driver of species interactions in the network.
	
	 The simulation of networks produced pairings where no interaction between species occurred (from now on called 'dead interactions'), which is an inherent problem resulting from the approach chosen by the authors. The distribution of dead interactions for all simulations can be seen in figs. \ref{dead_int_low} \& \ref{dead_int_high}. 

\begin{figure}[H]
	 \includegraphics[width = \textwidth]{/home/leon/Documents/Uni/M.sc/Master Thesis/Networks/models/plot_sink/hist_dead_int_lower}
	 \caption{\textbf{Number of dead interactions for the lower trophic level} The boxplots show the three scenarios of different community variable importances noted as Atl, aTl and, atL. Noted as org are the numbers of dead interactions for a default network without specific importances of the community variables}
	 \label{dead_int_low}
\end{figure}

\begin{figure}[H]
	 \includegraphics[width = \textwidth]{/home/leon/Documents/Uni/M.sc/Master Thesis/Networks/models/plot_sink/hist_dead_int_higher}
	 \caption{\textbf{Number of dead interactions for the higher trophic level} The boxplots show the three scenarios of different community variable importances noted as Atl, aTl and, atL. Noted as org are the numbers of dead interactions for a default network without specific importances of the community variables}
	 \label{dead_int_high}
\end{figure}

% by simulating networks problematic sampling effects are avoided (blüthgen2010, vazquez2022)


	% partner choice for rewiring based on values from init sim; using cophenetic.phylo fy for phylogenetic distance
	Partner choice in the rewiring process of the extinction simulations was based on the abundances, traits and phylogeny of the initial simulation. For phylogeny, the phylogenetic distances of the higher and lower trophic level were computed using the 'cophenetic.phylo' function from the ape package.
	\subsection{Extinction simulation} \label{sec:extc_sim}
	After networks were simulated, a modified version (see supplementary materials 'extc.alg') of \citeauthor{Vizentin-Bugoni2019}'s algorithm was used to simulate extinction. The algorithm used here is structured as follows (see fig. \ref{fig:extc_alg} for a visual overview). To easily distinguish the different scenarios the following naming convention was established. The trophic level from which species are removed during the extinction simulation as well as the coextinction threshold are noted in superscript, respectively. The rewiring method is noted in subscript. For example \[Atl^{lower, 0}_{Abundance}\], refers to a simulation where Abundance was the main driver of species interactions during the network simulation (see section \ref{sec:net_sim} for all five network abbreviations), species were removed from the lower trophic level during extinction simulation using a coextinction threshold of 0 \% and abundance based rewiring.

% put this in a box ?!	
	\begin{figure}[H]
	 \includegraphics[width = \textwidth]{/home/leon/Documents/Uni/M.sc/Master Thesis/Networks/approach_overview}
	 \caption{\textbf{Schematic overview of network \& extinction simulations} The left hand side shows the process of network simulation. From the initial network four netowrks were simulated according to the approach described in section \ref{sec:net_sim}. The right hand side shows the simplified structure of the extinction algorithm. The primary extinction (2.1) in each iteration is determined based on the user specified methods comprising trophic level (2.1.1) and species selection (2.1.2) (see section \ref{subsec:extc_alg} step \ref{itm:etxc} for details). Network A shows the species that is removed in black and the lost interactions with its partners in red. The algorithm then checks if other species are coextinct (2.2) by comparing the coextinction threshold to the amount of remaining interactions of each species (see section \ref{subsec:extc_alg} step \ref{itm:track} for details). Network B depicts the species lost due to coextinction in black with its interactions in orange. The next step is to determine interaction rewiring (2.3). Using one of the three rewiring methods (2.3.1, 2.3.2 or 2.3.3), species try to rewire to new partners or strengthen interactions with their other partners (see section \ref{subsec:extc_alg} step \ref{itm:rew} for details). Network C shows possible rewiring for one species (shown in black) based on abundance (red), traits (blue) or phylogeny (beige).
	 The last step of each iteration is redrawing the network from a multinomial distribution with the updated interaction probabilities (see section \ref{sec:net_sim} for details). The algorithm stops redrawing networks once only two species in the trophic level chosen for primary extinction are left. The remaining species are determined by running steps 2.1 - 2.3.}
	 \label{fig:extc_alg}
\end{figure}
	
\subsubsection{Extinction algorithm} \label{subsec:extc_alg}
\begin{enumerate} 
	\item \textbf{Delete dead interactions} {\small Check if dead interactions were introduced in the simulation process and delete them prior to extinction simulation.}
	\item \textbf{Set the adaptability of species} {\small This represents a probability which is used to express a species inclination to change interactions partners. The values range from 0 to 1 and are either set to 0.5 if this feature should not be used or originate from a uniform distribution.}
	\item \label{itm:etxc} \textbf{Extinction} {\small Delete a species from the network based on the provided methods (lower, higher or both trophic level(s) and random or abundance based selection, i.e. least abundant species first). This is executed using the extinction.mod function from \citeauthor{vizentin-bugoni2019} which is based on the extinction function from the bipartite package.}
	\item \label{itm:failsafe} \textbf{Handling dead interactions} {\small Since the algorithm redraws from a multinomial distribution using the updated interaction matrix to generate the network (see \ref{itm:redraw} for details), dead interactions can be introduced. This step is implemented to make sure the extinct species has valid interactions. If the extinct species only had dead interactions, the previous step is repeated until a species with valid interactions was found or three tries are reached. If this process is unsuccessful, the algorithm 'skips' the current extinction step by skipping steps (\ref{itm:rew}, \ref{itm:shift}, \ref{itm:track}, and \ref{itm:update}) and redraws a new network (i.e. continuing at step \ref{itm:redraw}). The simulation is aborted if there are three instances where steps were skipped because of dead interactions.}
	\item \label{itm:rew} \textbf{Rewiring} {\small All species that had at least one observed interaction with the extinct species, will try rewiring. The choice of the rewiring partner, i.e. the species that the afore mentioned species will try to rewire to, can be set by the user and comprises the following three options (all using the values from the initial simulation.)}
		\begin{itemize}
		\item \label{itm:abund_rew} \textbf{Abundance} {\small The species with the highest abundance is selected.}
		\item \label{itm:trait_rew} \textbf{Traits} {\small Euclidean distances of all traits are calculated. The species with the smallest trait distance across all traits compared to the extinct species is selected.}
		\item \label{itm:phylo_rew} \textbf{Phylogeny} {\small The species with the lowest phylogenetic distance to the extinct species is chosen. If multiple species have the same distance, one is selected at random.}
		\end{itemize}
	\item \label{itm:shift} \textbf{Preference shift} {\small Interaction probabilities are updated, simulating the process of species changing their foraging preferences due to losing one of their interaction partners. New interaction probabilities are the sum of (1) the values of the interaction between the extinct species and the rewiring partner and (2) the values of the interaction between the extinct species and the species that try to rewire. The latter is multiplied with the adaptability value of each species to account for their differences in how well or how willingly they rewire.}
	\item  \label{itm:track} \textbf{Tracking (co)extinctions} {\small Users can specify a threshold which determines coextinction of species based on the percentage of remaining interactions. If the quotient of total interactions after and before an extinction step is below or equal to the threshold, a species will be considered coextinct. Since the number of interactions are resimulated in each iteration, the divisor is the number of interactions of the last iteration and not the number of interactions of the initial web. Finally, the number of extinct and remaining species are noted for the higher and the lower trophic level.}
	\item \label{itm:update} \textbf{Update matrices} {\small The extinct species are deleted from the network and from the interaction matrix.}
	\item \label{itm:redraw} \textbf{Redraw network} {\small The updated interaction matrix is used to simulate observations by drawing from a multinomial distribution. This is the same step used in the data simulation (see \ref{sec:net_sim} for details) and can lead to the introduction of dead interactions to the network. Therefore dead interactions are registered and these species will remain in the network if they were 'inherited' from a previous redraw, i.e. extinction iteration.}
	\end{enumerate}
	% running extinction functions. using modified version of bugoni's extinction fy
		% algorithm structure:
			% delete dead interactions from webs
			% set adaptability (prob) of sp; either draw from unif dist or .5 for all
			% delete sp from web using extinction.mod from bugoni (based on bipartite)
			% dead interaction handling; retry max three times to find sp that has also valid interactions, if unsuccessful skip iteration
			% choice of rewiring partner (i.e. sp that shall replace extc sp; always compared to extc sp):
				% abund: use abundances from init sim to select sp with highest abund
				% trait: calculate euclidean distances of all traits from init sim; select sp with smallest trait distance over all distances
				% phylo: use sp. with lowest phylogenetic distance; if more than 1 sp have same distance randomly choose one
			% shift interaction probs; sum of interaction probs of rew partner and extc sp and interaction probs of extc sp and sp that try rewiring (i.e. old 					  interactions partners of extc sp) times the adaptability of each sp
			% df with number of extc sp & remaining sp per level is updated
			% update imat; retain dead interactions that were introduced by imat in previous step
			% calculate new web from updated interactions; check if dead interactions were introduced
			
		% following extinctions were run (always for lower and higher level extc, except for 'both'; also w/ and w/o rewiring):
			% org = default com importance; random extinction of sp
			% abund = abundance led extinction of sp
			% both = random extc from lower or higher level, random extc of sp
			% random = random extc of sp
 Four distinct extinction scenarios were simulated. (1) Default importances of the community variables and choosing a random species for extinction, (2) Abundance, Trait or Phylogeny based networks with random choice of the extinct species, (3) Abundance, Trait or Phylogeny based networks with abundance based choice of the extinct species, (4) Abundance, Trait or Phylogeny based networks with random extinction on one of either trophic level and random choice of the extinct species.
 Further more each of the four simulation scenarios were run with the coextinction threshold set to 0, 25, 50, or 75 \%. 
 
\begin{table}[H]
\centering
\caption{\textbf{Overview of extinction models} If community variables were set to low or high contribution to determine species interactions (see \ref{sec:net_sim} for details) one model with each setting was simulated. For trophic levels either the higher or lower level was chosen and extinctions were simulated. In the 'Trophic' model either the lower or higher trophic level was chosen randomly each extinction step and a species was removed. Except for the 'Abundance' model, the species that was removed was chosen at random. For the 'Abundance' model the species with the lowest abundance was removed sequentially. Each model was run with and without rewiring.}
\label{tab:models}
\resizebox{\textwidth}{!}{%
\begin{tabular}{llll}
Model & Community variables & Trophic levels & Species removed \\ \hline
Null & None & lower; higher & Random \\
Base & Abundance; Traits; Phylogeny & lower; higher & Random \\
Trophic & Abundance; Traits; Phylogeny & lower \& higher & Random \\
Abundance & Abundance; Traits; Phylogeny & lower; higher & Abundance based
\end{tabular}%
}
\end{table}
 
 
	% computing means of all simulations for each scenario
	The individual runs of each extinction scenario led to different lengths of the extinction sequences, therefore the longest sequence of each of the 10 simulations per web was chosen and the simulations with shorter sequences were filled with zeros to match their lengths. Then the mean per web was calculated.
	% calculating percentages of remaining sp
	With the mean number of extinct and remaining species per step of the extinction sequence, the percentage of remaining species was calculated.
	% computing means of all webs
	Finally, the mean over all webs was calculated. Since the extinction scenarios also produced different lengths, the longest extinction sequence per scenario was used as the target length and shorter sequences were padded with zeros to match the target length.
\section{Results}

The networks that were simulated with abundance as driving force for interactions showed the highest H2' mean values ($\overline{H2'}\textsubscript{Atl} = 0.348$). Phylogeny based networks showed the second highest mean H2' values($\overline{H2'}\textsubscript{atL} = 0.281$), followed by networks where no community variable was regarded of particular importance($\overline{H2'}\textsubscript{atl}$ = 0.229). Trait based networks showed second lowest mean H2' values ($\overline{H2'}\textsubscript{Atl} = 0.197$) and networks with all community variables set to be important had the lowest interaction specialisation ($\overline{H2'}\textsubscript{Atl} = 0.1$). Figure \ref{fig:h2} provides an overview of H2' values of the five networks.
\begin{figure}[H]
	 \includegraphics[width = \textwidth]{/home/leon/Documents/Uni/M.sc/Master Thesis/Networks/models/plot_sink/h2_all}
	 \caption{\textbf{Two dimensional Shannon Entropy of all networks} The boxplots show all scenarios of different community variable importances noted as Atl, aTl, atL, and, ATL. Noted as atl are values for a default network without specific importances of the community variables}
	 \label{fig:h2}
\end{figure}


The highest mean connectance was observed in networks where all community variables were set to contribute with high importance ($\overline{C}\textsubscript{ATL} = 0.796$). Networks where the influence of community variables were left unchanged had the second highest mean connectance ($\overline{C}\textsubscript{atl} = 0.32$), followed by networks with abundance as their main driver of interaction between species ($\overline{C}\textsubscript{Atl} = 0.3$). Trait based and phylogeny based networks showed the lowest mean connectance with $\overline{C}\textsubscript{aTl} = 0.217$ and $\overline{C}\textsubscript{atL} = 0.156$, respectively. See figure \ref{fig:cntc} for an overview of all networks.
\begin{figure}[H]
	 \includegraphics[width = \textwidth]{/home/leon/Documents/Uni/M.sc/Master Thesis/Networks/models/plot_sink/cntc_all}
	 \caption{\textbf{Connectance of all networks} The boxplots show all scenarios of different community variable importances noted as Atl, aTl, atL, and, ATL. Noted as atl are values for a default network without specific importances of the community variables}
	 \label{fig:cntc}
\end{figure}

% validate AUC values !!!

% Atl
Secondary extinctions were lowest in networks where abundance was the main driver of interactions regardless of the rewiring method and whether species from the lower or upper trophic level were removed. 
% Phylogeny
Phylogeny based rewiring mitigated the propagation of extinctions more effectively than abundance and trait based rewiring. It also showed a higher network robustness compared to the extinction simulation without rewiring when species from the lower trophic level were removed ($AUC\textsubscript{lower; phylo, Atl} = $, $AUC\textsubscript{lower; norew, Atl} = $, see figure \ref{fig:extc_sims_lower} g). When removing species from the higher trophic level phylogeny based rewiring also led to higher network robustness than without rewiring
% Trait

Rewiring led to higher robustness in each simulation except for trait and phylogeny based networks using abundance as rewiring method.


% Column wise; by rewiring
Among the three rewiring methods, phylogeny based rewiring resulted in the highest robustness regardless of coextinction threshold, if species from the lower or higher level were removed and which community variables were used to determine species interactions. Simulations with trait based rewiring yielded marginally lower or equal robustness compared to phylogenetic rewiring. Abundance based rewiring resulted in the lowest robustness compared to trait and phylogeny based rewiring.

% Row wise; by com var
Comparing the four scenarios that were used to determine species interactions, simulations with a high contribution of abundance always showed the highest robustness. Simulations using the original networks resulted in the second highest network robustness. Trait based networks 


% In both cases, when species from the lower level were removed, the coextinction curves showed a slight decline in the beginning, followed by a substantial loss of species after a certain threshold.

Phylogenetic rewiring also showed higher robustness than simulations without rewiring. With a coextinction threshold of 100 \% simulations with and without rewiring did not differ meaningfully.
%, with one exception. In simulations where the coextinction threshold was set to 75 \%, phylogeny had a high influence on species interactions and species were removed from the lower trophic level, Abundance based rewiring showed the highest robustness among rewiring methods.

\begin{figure}[H]
	 \includegraphics[width = \textwidth]{/home/leon/Documents/Uni/M.sc/Master Thesis/Networks/models/plot_sink/extinction_cascade_lower_trophic_level_NULL}
	 \caption{\textbf{Mean extinction cascades} The plots show the mean loss of higher trophic level species in response to the extinction of lower trophic level species for each network and rewiring method. The rows represent the rewiring methods (abundance, trait, and phylogeny; see section \ref{subsec:extc_alg} step \ref{itm:rew} for details) from top to bottom, respectively. Columns represent the different networks (Atl, aTl, and atL; see section \ref{sec:net_sim} for a detailed explanation of abbreviations). The solid red line in each plot indicates the mean loss of species with rewiring. The solid grey line is the mean of extinctions without rewiring. Dotted lines mark the 95 \% confidence interval.}
	 \label{fig:extc_sims_lower}
\end{figure}

\begin{figure}[H]
	 \includegraphics[width = \textwidth]{/home/leon/Documents/Uni/M.sc/Master Thesis/Networks/models/plot_sink/extinction_cascade_higher_trophic_level_NULL}
	 \caption{\textbf{Mean extinction cascades} The plots show the mean loss of lower trophic level species in response to the extinction of higher trophic level species for each network and rewiring method. The rows represent the rewiring methods (abundance, trait, and phylogeny; see section \ref{subsec:extc_alg} step \ref{itm:rew} for details) from top to bottom, respectively. Columns represent the different networks (Atl, aTl, and atL; see section \ref{sec:net_sim} for a detailed explanation of abbreviations). The solid red line in each plot indicates the mean loss of species with rewiring. The solid grey line is the mean of extinctions without rewiring. Dotted lines mark the 95 \% confidence interval.}
	 \label{fig:extc_sims_higher}
\end{figure}


\newpage
	\section{Discussion}
% Shortly repeat the Question(s)

% Summarize the main findings

% sampling biases; vazquez2022, blüthgen2010
% Using connectance as a metric is unproblematic here since networks are simulated and therefore sampling effort is not influencing the number of detected interactions
% connectance & connectivity vary sig. between biogeographic regions (oelsen&jordano2002) but majority of sampled networks are coming from south america (e.g. brasil), apparently little data for temperate climate (i.e. europe or north america)
In addition to the mentioned sampling biases, there also seems to be a bias in sampling location. In the case of plant-pollinator networks, South America is clearly overrepresented. For example, half of the available data on plant-pollinator networks on the Interaction Web Database (\url{http://www.ecologia.ib.usp.br/iwdb/index.html}) come from Brazil, Argentina, Chile, Venezuela, and Ecuador. This might have influenced our common understanding of interactions networks since connectance and connectivity were shown to significantly vary between biogeographic regions \parencite{Oelsen2002}.


% see rezende2007 for phylogenetical rewiring !!


% only "quantitative" change in com_vars, could be interesting to change "qualitatively" i.e. abundance distribution (log-normal, normal, etc.), phylogeny (branching-early / branching late)...
\section{Conclusion}
\section{Acknowledgments}
\newpage
\section*{Declaration of Originality}
I, Leon Thoma hereby declare that the presented thesis is my own work and that I have not sought or used inadmissible help of third parties to produce this work. Furthermore, I have clearly referenced all sources used in the work and used inverted commas for all text directly or indirectly quoted from a source.\paragraph{}
This work has not been submitted to another examination institution - neither in Germany nor outside Germany - neither in the same nor in a similar way and has not been published.\paragraph{}

Freiburg,\paragraph{}

\rule{5cm}{.4pt}\par
Leon Thoma
\newpage
\begin{multicols}{2}[\printbibheading]
\printbibliography[heading=none]
\end{multicols}
\end{document}