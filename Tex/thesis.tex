\documentclass[12pt,a4paper]{article}
\usepackage[utf8]{inputenc}
\usepackage{csquotes}
\usepackage[english]{babel}
\usepackage[T1]{fontenc}
\usepackage{graphicx}
\usepackage{float}
\author{Leon Thoma}
\title{Master thesis in environmental sciences}
\usepackage[backend=biber, style=authoryear, giveninits=true]{biblatex}%  !!!In bib only state the first char of the first name!!!
\AtBeginBibliography{\footnotesize}
\usepackage{url}
\usepackage[font={footnotesize}, labelfont=bf]{caption}
\usepackage[table,xcdraw]{xcolor}
\usepackage{caption}
\usepackage{amssymb}
\usepackage{amsmath}
\addbibresource{extc_net.bib}
\defbibheading{none}{}
\usepackage[colorlinks]{hyperref}
\hypersetup{
    colorlinks,
    linkcolor={black!50!black},
    citecolor={black!50!black},
    urlcolor={blue!80!black}
}
\usepackage{multicol}
\usepackage[margin=1in]{geometry}
\usepackage{tabularx,booktabs}
\usepackage{lscape}
\usepackage{rotating}

\begin{document}
\begin{titlepage}
	\centering
	\includegraphics[width=0.25\textwidth]{logo-grundversion-01}\par\vspace{1cm}
	{\scshape\large Albert-Ludwigs-Universität Freiburg\par}
	\vspace{1.25cm}
	{\scshape\large Thesis for the attainment of the degree:\\ Master of science\par}
	\vspace{.75cm}
	{\scshape\large Faculty for Environment and Natural Resources\par}
	\vspace{.75cm}
	{\scshape\large Chair of Biometry and Ecosystem Analysis\par}
	\vspace{.75cm}
	{\Large\itshape Rewiring networks under species loss: a simulation
study
\par}
	\vspace{.75cm}
	{\Large\itshape Leon Thoma\par}
	\vspace{.25cm}
	{\scshape\normalsize E-mail: leonthoma@gmx.de\par}
	\vspace{.15cm}
	{\scshape\normalsize Matriculation number: 3733253\par}
	\vspace{.75cm}
	\large Examiner:\par
	\large Prof. Dr. Carsten  \textsc{Dormann} \par
	\vspace{.5cm}
	\large Co-Examiner:\par
	\large Dr. Anne-Christine \textsc{Mupepele}
	\vfill

	{\large \today\par}
\end{titlepage}
	\tableofcontents
	\newpage	
	\section{Abstract} 

\newpage%
%
% RECHECKING:
% GO THROUGH ALL CITATIONS IN JABREF FOR CORRECT SPECIAL CHARACTERS AND GENERAL CHECK
%
\section{Introduction}
% Content
% Short intro to networks
% Drivers of species interaction
	% Neutral theory
	% Niche theory
	% Spatio-temporal variability
% Intro to extinction simulation
	% What has been done
	% State of knowledge
%	-> connection to gap of knowledge
% Gap of knowledge/Approach
	% Why ?
	% Reserch questions





Contrary to a main question in the field of invasion ecology, how novel species influence interactions among species, stands the question of how species leaving an ecosystem affects the remaining species. 

% species leaving network can have influence on connection between sp over different biomes/ ecosystems (terrestrial/aquatic; knight 2005 in colwell 2012)
% coextinction is difficult to document (colwell 2012):
	% determining host specifity -> interaction partners might become extinct before their hosts (Moir 2010, 2012b, Powell 2011)
	% if spcies are reintroduced to wild, interactions partners might be affected by treatment used to bring up reintroduced species (e.g. biocides)
	% 

%
%The introduction of species was studied in a wide variety of ecosystems (references here) and across many species (references here), studying the effects of extinctions, however is much harder. This is due to the fact that  
% Intro to topic:
	% network ecology
		% lau2017
		% poisot2016; 10.1111/1365-2435.12799
		% vazquez2022
\subsection{Network ecology}
Network ecology has helped researchers to develop a more detailed understanding of ecological communities. From analysing structures within networks and identifying patterns \parencite{Jordano1987, Dunne2002 }, % Olesen2007, Bascompte2003 
to finding the underlying processes that lead to the observed patterns \parencite{Rezende2007, Vazquez2009, Thebault2010}.

% relation between network properties/structure  https://doi.org/10.1111/j.1365-2656.2010.01688.x 
%Ecological communities are subject to constant change, due to a variety of reasons such as habitat fragmentation, global warming, introduction of species or loss thereof, to name some \parencite{}. 


\subsubsection{Species interactions}
A major part of understanding the formation, upkeep and remodelling that is observed in ecological communities, is how interactions between species are formed as well as how and why the change. Historically, researchers tried to explain interactions with two main theories. Niche theory based the reasoning on biological information such as morphology, phenology, behaviour, evolutionary history and phylogeny \parencite{Webb2002, Jordano2003, Vazquez2009}. % all from vazquez2022
Consequently, the probability of encounter is shaped by the traits of species.
% Some possible links are unrealized because of the mismatch in niche space (for example due to missing phenological overlap)\parencite{Canard2012}.
The other approach that was developed, called neutral theory, argues that species are equivalent in their ecological role and interactions are the result of random encounters. Therefore, interactions in communities should mostly be shaped by abundance \parencite{Vazquez2005, Vazquez2007}.\\
There are studies providing evidence that processes from both theories are underlying species interactions \parencite{Vazquez2009b, Verdu2011}. Further research has shown that the importance of the proposed drivers vary, for example that phenological overlap and trait matching are more important than abundance \parencite{Vizentin-Bugoni2014}, that native and exotic species are affected differently by niche processes \parencite{Peralta2020} or that they vary along a latitudinal gradient \parencite{Sonne2020}. Thus most studies focused on predicting interactions between species, combined processes of neutral and niche theory, most commonly including traits, phenology, phylogeny and abundance as variables \parencite{Pearse2013, Olito2014, Morente-Lopez2018, Kotula2021, Benadi}.

 \par 
More recently however, the effect of temporal and spatial variation on interactions and network structure moved into focus \parencite{Schwarz2020, CaraDonna2021}. \citeauthor{Olesen2008} showed that interactions varied in plant-pollinator networks due to temporal and spatial turnover \parencite{Olesen2008}. \citeauthor{CaraDonna2017} found that temporal dynamic in networks was high, and especially that species frequently changed interaction partners \parencite{CaraDonna2017}.

% for trait see bartomeus2018

% Not really interested in influence of choice method or removal sequence on robustness. rather, how do a, t, l influence networks, e.g. structures (h2', connectance, etc.), interactions or robustness. also if a ,t or l were used to determine rewiring partner what were the results ? 

In the face of the current rate of extinctions (e.g. see \cite{IPBES}), it is crucial to develop a deeper understanding of the changes an extinction event induces in ecological communities, in order to preserve it's function and prevent extinction cascades. 
\subsection{Extinction simulation}
	% Different approaches (Memmot, Kaiser-Bunbury, Vieira, Baumgartner, Vizentin-Bugoni)
The following 3 models mark cornerstones in advances of extinction cascade modelling. In 2004 \citeauthor{Memmott2004} simulated secondary extinction for plant-pollinator networks using three scenarios for initial extinctions of pollinators (but not plants). Random extinction (Null model), sequential removal of species starting with the most linked, and sequential removal starting with the least linked species. A species was considered to be coextinct if it lost all its interactions. However, the approach disregarded a few key points, such as the quality of a given interaction (i.e. pollination success and frequency of visits), the fact that plants can reproduce even without pollinators e.g. by selfing and arguably most important that species don't compensate for lost interaction partners (either by creating new interactions or enhancing existing ones). Nevertheless, the research of \citeauthor{Dunne2002} and \citeauthor{Memmott2004} provided helpful insights into network structure showing that species in less connected networks are more prone to suffer from coextinction. \par

A first quantitative approach to extinction modelling in mutualistic networks considering behavioural shifts (i.e. interaction rewiring) was done by \citeauthor{Kaiser-Bunbury2010} in 2010 \parencite{Kaiser-Bunbury2010}. They based rewiring on interactions between species that were observed at one point of time when the site was sampled. For example, if a species lost all it's interaction partners it would remain in the model with it's full interaction strength if a partner was present with which they observed an interaction in the network at an earlier or later point in time. Their results indicated that networks are more robust to extinction when rewiring is considered since it's 'artificially' increasing connectance. \par

In 2019 \citeauthor{Vizentin-Bugoni2019} extended the approach of \citeauthor{Memmott2004} and \citeauthor{Dunne2002} \parencite{Memmott2004, Dunne2002} for (co)extinction simulation.
% by enhancing the functions provided in the bipartite R-package \parencite{dormann2008}. 
%% Rephrase !!!
Rewiring was implemented as a two step process. First a potential partner is chosen based either on (1) variables known to be important for species interactions in the studied network (e.g. abundance) or (2) at random. Second, the model computes a rewiring probability for the chosen partner, accounting for species abundance, phenological overlap, morphological traits and combinations thereof. Finally, a binomial trail will determine if rewiring is successful or not. Furthermore, the model allows its users to choose the number of rewiring attempts \footnote{
One try with a single partner, 
multiple tries with a single partner, 
one try with each partner, 
multiple tries with multiple partners
}. Following the assumption of \citeauthor{Memmott2004} and \citeauthor{Dunne2002}, a species was considered coextinct if it lost all it's interaction partners.
\par


	% general results of extc sim w/ rewiring
		% robustness increases w/ rewiring: dunne2002, eklöf2006, thebault & fontaine2010, baumgartner2020
The results of research on extinctions and their propagation in ecological networks have shown, that loss of generalists causes a higher rate of coextinction than loss of specialists (\cite{Memmott2004, Kaiser-Bunbury2010, Traveset2017, Bastazini2018, Vizentin-Bugoni2019, Biella2020}; but see \cite{Dunne2002}). 
-> WHY

Another result was a substantially higher robustness in mutualistic bipartite networks, when species were removed from the higher trophic level compared to when lower trophic level species were removed \parencite{Schleuning2016}.

%This was the case irrespective of how the species that was to be removed was chosen, i.e. random, abundance based, etc.(RECHECK!!!) 
%This is most likely due to the fact, that species' roles in networks are highly asymmetric, meaning species belonging to the higher trophic level are very dependent on their interaction partners from the lower level, but not vice versa \parencite{Bascompte2006}.

\citeauthor{Dunne2002, Ekloef2006} and, \citeauthor{Thebault2010} found a positive correlation of network robustness and connectance \parencite{Dunne2002, Ekloef2006, Thebault2010}. However, \citeauthor{Vieira2015} showed that the opposite was the case \parencite{Vieira2015}. They argued, that the assumption, that coextinction only occurs when a species loses all its interaction partners, which the previous studies made, led to their finding. Recently, \citeauthor{Baumgartner2020}, challenged the reasoning of \citeauthor{Vieira2015} by showing a positive correlation between network robustness and connectance using an extinction algorithm that accounts for the interaction strengths (i.e. dependence), intrinsic demographic dependence, and dissimilarity of species to estimate coextinction susceptibility \parencite{Baumgartner2020}. 
% by including coext threshold this can be tested here !!! 


\parencite{Kaiser-Bunbury2010, Schleuning2016, Timoteo2016, Costa2018} higher robustness due to rewiring. % also (Ramos‐Jiliberto et al.,
%2012; Nuwagaba et al., 2017) from baumgartner2020; compare with blüthgen2010 and vieira2015


% Considering this...
% i will use....
% to find out that...

% Explaining problem/ gap of knowledge
Despite the insights previous studies provided, it is still unclear if robustness is influenced by how species interact (ie. what drives the interactions) and find new partners to substitute lost interactions with extinct partners. To provide further insight into this possible interrelationship, this study will focus on modelling extinction cascades using simulated networks controlling the influence of variables found to predict species interactions.\\

% For example, using the same rewiring criteria, does robustness change when the main driver of interactions in the network is changed ? Its still unknown if e.g. the abundance of species in a network changes the propagation of extinctions. 
	% rewiring methods
%	not clear how different rewiring mechanisms affect extinction propagation. especially in bugoni et al combining rewiring mechanisms led to lower effect size than single rewiring mechanisms

% Explain the approach
The analysis of this study was a two step process. First, networks were simulated and then used in the extinction algorithm to assess network robustnesses (see fig. \ref{fig:extc_alg}).  
Network simulations were based on the approach presented by \citeauthor{Benadi} \parencite{Benadi}. The 'observed' interactions in a network are simulated by combining abundance, trait, and phylogenetic information (also referred to as community variables). The influence of the three variables on the formation of interactions between species was altered by using either one, none or all of them as the main driver. Additionally, a group was added, where the importances of the three variables mentioned above were not changed. Thus, six ways of species interaction simulations were implemented.
To identify differences in the network structures among the species interaction scenarios, the two dimensional shannon entropy (H2') was calculated as a measure for specialization \parencite{Bluethgen2006}.



%	simulating networks has the advantage of equal "sampling effort" and is therefore not affected by sampling biases !!
	
%The constant 're-observation' of species interactions during the extinction simulations functions as a simple proxy of spatio-temporal variability of networks. 
% Research questions
%Does network robustness change for different rewiring methods when the driving force of interactions in networks is altered ?
%Is network robustness affected by underlying interrelations between the drivers of species interaction and their way of choosing rewiring partners 
% higher connectance leads to higher robustness (following dunne2002, eklöf2006, thebault & fontaine 2010, baumgartner2020
The questions this study aims to answer are:
\begin{enumerate}
\item Is network robustness influenced by the way species interactions are simulated ?
\item Does the choice of rewiring method lead to different network robustnesses ?
\end{enumerate}

% "third question"
%The interrelationship between species interaction scenarios and rewiring method was not part of the main questions of this study, however, since there seemed to be some interaction between the two, the relation of robustness and the percentage of species with only one interaction partner among rewiring methods was compared.

For the first question only simulations where no rewiring was used were considered. This allowed for unbiased comparisons between the species interaction scenarios. In order to compare rewiring methods, the extinction simulations were run only with networks where community variables were left unchanged.



%first question: what influence does com_vars have on robustness (compare all com_vars using norew)
%	-> 
%second question: what influence does rew have on robustness (compare all rew using org networks) -> focus of thesis should be rew methods !
%	-> rew should be most efficient for specialised networks
%	
%resulting third question: is there an interrelationship of com_vars and rew that influences robustness (use no of singletons per network comparing rew methods)
%	-> comparing H2' would expect highest effect of rewiring for specialised networks, but maybe only highly abundant/trait matching/phylogeny matching species are specialised and the rest isn't. therefore check no of singeltons in networks (ie sp most endangered by coextinction) and compare the robustness 

%However, since phylogenetically closely related species often show similar traits, the set of species they interact with also tends to overlap \parencite{Gomez2010, Rezende2007}. The difference in robustness for trait and phylogeny based networks should thus be small when using trait or phylogeny based rewiring. 

	\section{Material \& Methods}
	The simulations in this study were executed in R \parencite{Rcore} (version 3.6.3) based on the functions of the tapnet\footnote{The paper is yet to be published, but the package and it's vignette are available via CRAN and GitHub; https://cran.r-project.org/web/packages/tapnet/index.html, https://github.com/biometry/tapnet} \parencite{Benadi} and bipartite \parencite{Dormann2008} packages as well as the approach of the extinction algorithm developed by \citeauthor{Vizentin-Bugoni2019} \parencite{Vizentin-Bugoni2019}. Some of the functions provided by these sources were modified or augmented to match the specific needs of the presented approach. The source code of this analysis is available as a supplementary file. A schematic overview of the network and extinction simulations of this study is presented in the figure below.	
	
	\begin{figure}[H]
	 \includegraphics[width = \textwidth]{/home/leon/Documents/Uni/M.sc/Master Thesis/Networks/approach_overview}
	 \caption{\textbf{Schematic overview of network \& extinction simulations} The left hand side shows the process of network simulation. From the initial network five networks were simulated according to the approach described in section \ref{sec:net_sim}. The right hand side shows the simplified structure of the extinction algorithm. The primary extinction (2.1) in each iteration is determined based on the user specified methods comprising trophic level (2.1.1) and species selection (2.1.2) (see section \ref{subsec:extc_alg} step \ref{itm:etxc} for details). Network A shows the species that is removed in black and the lost interactions with its partners in red. The algorithm then checks if other species are coextinct (2.2) by comparing the coextinction threshold to the amount of remaining interactions of each species (see section \ref{subsec:extc_alg} step \ref{itm:track} for details). Network B depicts the species lost due to coextinction in black with its interactions in orange. The next step is to determine interaction rewiring (2.3). Using one of the six rewiring methods (2.3.1 to 2.3.6), species try to rewire to new partners or strengthen interactions with their other partners (see section \ref{subsec:extc_alg} step \ref{itm:rew} for details). Network C shows possible rewiring for one species (shown in black) based on abundance (red), traits (blue) or phylogeny (beige).
	 The last step of each iteration is redrawing the network from a multinomial distribution with the updated interaction probabilities (see section \ref{sec:net_sim} for details). The algorithm stops redrawing networks once only two species in the trophic level chosen for primary extinction are left. The remaining species are determined by running steps 2.1 - 2.3.}
	 \label{fig:extc_alg}
\end{figure}
		\par


% describe general approach:
	% initial network sim
	% sim webs w/ com_var
	% choose rewiring method & extinction method
	% run extc models

	\subsection{Network simulation} \label{sec:net_sim}
To simulate the networks, each of the three community variables (abundance (a), traits (t), and phylogeny/latent traits (l)) were provided as an independent two-dimensional matrix defined by the species of the two trophic levels. The algorithm first derives interaction frequencies from the linear combination of the three community variable matrices, which are all scaled to sum of one before and after their multiplication. Interaction frequencies are then used as probabilities in a multinomial draw to calculate the desired number of 'observed' interactions (see also fig. 1 in \cite{Benadi} for a conceptual overview).	 The community variable matrices were the same for all networks and were inherited from an initial network simulation with the default tapnet algorithm. The number of species in the trophic levels of the initial network was 40 and 50 for the lower and higher level, respectively. Total number of observations was set to 1111. The number of simulated traits was set to 2 for phylogenetically correlated and uncorrelated traits each.\par
	% caradonna: nobs: 30000, ntraits: 2, nlow: , nhigh:
	% kaiser-bunbury: nobs: 1278, ntraits: NA, nlow: , nhigh:
	% costa: nobs: 	3974, ntraits: NA, nlow: , nhigh:
	% bugoni: nobs: na, ntraits; 2, nsims:1000, nlow: , nhigh
	% schleuning:   see supp mat.
	% memmot (10.1046/j.1461-0248.1999.00087.x): nobs: 2183, nlow: , nhigh: 
	% vazquez/simberloff (10.1046/j.1461-0248.2003.00534.x): nobs: 5285, nlow: , nhigh:
	

	% using simulated abundances, traits and other params (paramlist) simulate omat for all com importance scenarios
	Using the abundance \& trait matrices and the phylogenetic eigenmatrices from the initial simulation, 1000 networks were simulated for each of the six species interaction scenarios. This was achieved by exponentiating the corresponding matrix of each scenario with either 0.1 or 1.9, resulting in high or low importance respectively. For the inital networks none of the variables were regarded as especially important, matrices were thus left unchanged (i.e. the exponent was 1). The scenarios will be referred to as Atl, aTl, atL, atl, and ATL, with capital letter(s) indicating high importance and lower letters indicating low importance of the respective community variable. Furthermore, the simulations of the inital network are called original.\par
	
	 The simulation of networks produced pairings where no interaction between species occurred (from now on called 'improbable interactions'). This is due to interaction probabilities being so small that an interaction is not 'observed' every time the network is simulated. Therefore, species pairs with improbable interactions were removed before using the networks for extinction simulation. However, since marginal totals were used to identify and delete species pairs with improbable interactions, there were cases where after removing a species, other pairings had no observed interactions. For example, if a species had only one interaction with another species that was deleted, the first species was then left without interactions. Because of this, the actual number of species per web was not equal to the number specified during network simulation. The distribution of improbable interactions for all networks can be seen in figure \ref{fig:dead_int}.

% Improbable interactions
\begin{figure}[H]
	 \includegraphics[width = \textwidth]{/home/leon/Documents/Uni/M.sc/Master Thesis/Networks/models/plot_sink/dead_pre_both}
	 \caption{\textbf{Number of improbable interactions} The boxplots show the number of improbable interactions for each community variable importance scenario.
	 The left plot (A) shows species from the lower trophic level, the right plot (B) shows species from the higher trophic level. See section \ref{sec:net_sim} for a definition of improbable interactions and a detailed explanation of the abbreviations of community variable importance scenarios.}
	 \label{fig:dead_int}
\end{figure}

% by simulating networks problematic sampling effects are avoided (blüthgen2010, vazquez2022)

	\subsection{Extinction simulation} \label{sec:extc_sim}
	After networks were simulated, a modified version (see supplementary materials 'extc.alg') of \citeauthor{Vizentin-Bugoni2019}'s algorithm was used to simulate extinction. The algorithm used here is structured as follows (see fig. \ref{fig:extc_alg} for a visual overview). 
%To easily distinguish the different scenarios, the following naming convention was established. The trophic level from which species are removed during the extinction simulation as well as the rewiring method are noted in superscript, respecitvely. For example Atl\textsubscript{lower, abund}, refers to a simulation where Abundance was the main driver of species interactions during the network simulation (see section \ref{sec:net_sim} for network abbreviations), species were removed from the lower trophic level during extinction simulation with abundance based rewiring.

\subsubsection{Extinction algorithm} \label{subsec:extc_alg}
\begin{enumerate} 
	\item \textbf{Delete improbable interactions} {\small Check if improbable interactions were introduced in the simulation process and delete them prior to extinction simulation.}
	\item \textbf{Set the adaptability of species} {\small This represents a probability which is used to express a species inclination to change interactions partners. The values range from 0 to 1 and are either set to 0.5 if this feature should not be used or originate from a uniform distribution.}
	\item \label{itm:etxc} \textbf{Extinction} {\small Delete a species from the network based on the provided methods (lower or higher trophic level).}
%This is executed using the extinction.mod function from \citeauthor{vizentin-bugoni2019} which is based on the extinction function from the bipartite package.}
	\item \label{itm:failsafe} \textbf{Handling improbable interactions} {\small Since the algorithm redraws from a multinomial distribution using the updated interaction matrix to generate the network (see step \ref{itm:redraw} for details), improbable interactions can be introduced. This step is implemented to make sure the extinct species has valid interactions. If the extinct species has no interactions, the previous step is repeated until a species with valid interactions was found or three tries are reached. If this process is unsuccessful, the algorithm 'skips' the current extinction iteration by skipping steps (\ref{itm:rew}, \ref{itm:shift}, \ref{itm:track}, and \ref{itm:update}) and redraws a new network (i.e. continuing at step \ref{itm:redraw}). The simulation is stopped if there are three instances where iterations were skipped because of improbable interactions. This is an optional feature and users can decide if they want to use it or not.}
	\item \label{itm:rew} \textbf{Rewiring} {\small All species that had at least one observed interaction with the extinct species, will try rewiring. The choice of the rewiring partner, i.e. the species that the afore mentioned species will try to rewire to, can be set by the user and comprises the following five options (all using the values from the initial simulation)}
		\begin{itemize}
		\item \label{itm:abund_rew} \textbf{Abundance} {\small The species with the highest abundance is selected.}
		\item \label{itm:trait_rew} \textbf{Traits} {\small Euclidean distances of all traits are calculated. The species with the smallest trait distance across all traits compared to the extinct species is selected.}
		\item \label{itm:phylo_rew} \textbf{Phylogeny} {\small The species with the lowest phylogenetic distance to the extinct species is chosen. If multiple species have the same distance, one is selected at random.}
		\item \label{itm:AT_rew} \textbf{Abundance x Trait} {\small The rewiring probabilities were calculated according to the respective method and their sum was used}
		\item \label{itm:AP_rew} \textbf{Abundance x Phylogeny} {\small The rewiring probabilities were calculated according to the respective method and their sum was used}
		\end{itemize}
	\item \label{itm:shift} \textbf{Preference shift} {\small Interaction probabilities are updated, simulating the process of species changing their foraging preferences due to losing one of their interaction partners. New interaction probabilities are the sum of (1) the values of the interaction between the extinct species and the rewiring partner and (2) the values of the interaction between the extinct species and the species that they try to rewire. The latter is multiplied with the adaptability value of each species to account for the differences in how well or how willingly they rewire. If no adaptability values are provided, the default vaules was arbitrarily set to 0.5}
	\item  \label{itm:track} \textbf{Tracking (co)extinctions} {\small Users can specify a threshold which determines coextinction of species based on the percentage of remaining interactions. If the quotient of total interactions after and before an extinction step is below or equal to the threshold, a species will be considered coextinct. Since the number of interactions are resimulated in each iteration, the divisor is the number of interactions of the last iteration and not the number of interactions of the web provided initially. Finally, the number of extinct and remaining species are noted for the higher and the lower trophic level.}
	\item \label{itm:update} \textbf{Update matrices} {\small The extinct species are deleted from the network and from the interaction matrix.}
	\item \label{itm:redraw} \textbf{Redraw network} {\small The updated interaction matrix is used to simulate observations by drawing from a multinomial distribution. This is the same step used in the data simulation (see \ref{sec:net_sim} for details) and can lead to the introduction of improbable interactions to the network. Therefore these species will remain in the network if they were 'inherited' from a previous redraw, i.e. extinction iteration, because they would otherwise be falsely considered extinct.}
	\end{enumerate}
	% running extinction functions. using modified version of bugoni's extinction fy
		% algorithm structure:
			% delete dead interactions from webs
			% set adaptability (prob) of sp; either draw from unif dist or .5 for all
			% delete sp from web using extinction.mod from bugoni (based on bipartite)
			% dead interaction handling; retry max three times to find sp that has also valid interactions, if unsuccessful skip iteration
			% choice of rewiring partner (i.e. sp that shall replace extc sp; always compared to extc sp):
				% abund: use abundances from init sim to select sp with highest abund
				% trait: calculate euclidean distances of all traits from init sim; select sp with smallest trait distance over all distances
				% phylo: use sp. with lowest phylogenetic distance; if more than 1 sp have same distance randomly choose one
			% shift interaction probs; sum of interaction probs of rew partner and extc sp and interaction probs of extc sp and sp that try rewiring (i.e. old 					  interactions partners of extc sp) times the adaptability of each sp
			% df with number of extc sp & remaining sp per level is updated
			% update imat; retain dead interactions that were introduced by imat in previous step
			% calculate new web from updated interactions; check if dead interactions were introduced
			
		% following extinctions were run (always for lower and higher level extc, except for 'both'; also w/ and w/o rewiring):
			% org = default com importance; random extinction of sp
			% abund = abundance led extinction of sp
			% both = random extc from lower or higher level, random extc of sp
			% random = random extc of sp
				% partner choice for rewiring based on values from init sim; using cophenetic.phylo fy for phylogenetic distance
Partner choice in the rewiring process of the extinction simulations was based on the abundances, traits and phylogeny of the initial simulation. For phylogeny, the phylogenetic distances of the higher and lower trophic level were computed. Two extinction scenarios were simulated with 10 runs for each network. (1) removing species from the lower trophic level and (2) removing species from the higher trophic level. The coextinction threshold was set to 0 \%, i.e. a species was coextinct when it lost all interaction partners. Adaptability of all species was set to 0.5, and the algorithm was allowed to terminate simulations early (see step \ref{itm:failsafe})
 
Since the premature termination of the extinction algorithm due to improbable interactions could affect the results, simulations were repeated without this option using the same variables and approach. This was only done to preclude unintended effects, and results will therefore, only be presented in the supplemental materials.
%\begin{table}[H]
%\centering
%\caption{\textbf{Overview of extinction models} If community variables were set to low or high contribution to determine species interactions (see \ref{sec:net_sim} for details) one model with each setting was simulated. For trophic levels either the higher or lower level was chosen and extinctions were simulated. In the 'Trophic' model either the lower or higher trophic level was chosen randomly each extinction step and a species was removed. Except for the 'Abundance' model, the species that was removed was chosen at random. For the 'Abundance' model the species with the lowest abundance was removed sequentially. Each model was run with and without rewiring.}
%\label{tab:models}
%\resizebox{\textwidth}{!}{%
%\begin{tabular}{llll}
%Model & Community variables & Trophic levels & Species removed \\ \hline
%Null & None & lower; higher & Random \\
%Base & Abundance; Traits; Phylogeny & lower; higher & Random \\
%Trophic & Abundance; Traits; Phylogeny & lower \& higher & Random \\
%Abundance & Abundance; Traits; Phylogeny & lower; higher & Abundance based
%\end{tabular}%
%}
%\end{table}


Following the extinction simulations, the robustness of each network was calculated. To assess the effects of rewiring and network simulation on robustness a ANOVA was used with the number of species from the lower and the upper trophic level, community variable importance, rewiring method, $H2'$, the interaction between rewiring method and $H2'$, as well as the extinction scenario\footnote{i.e. removing species from the lower or from the higher trophic level} as predictors.
%Additional ANOVAs were used to see how the effects changed when the data was partitioned by species removal, network simulation scenarios, and rewiring methods.

 
%	% computing means of all simulations for each scenario
%	The individual runs of each extinction scenario led to different lengths of the extinction sequences, therefore the longest sequence of each of the 10 simulations per web was chosen and the simulations with shorter sequences were filled with zeros to match their lengths. Then the mean per web was calculated.
%	% calculating percentages of remaining sp
%	With the mean number of extinct and remaining species per step of the extinction sequence, the percentage of remaining species was calculated.
%	% computing means of all webs
%	Finally, the mean over all webs was calculated. Since the extinction scenarios also produced different lengths, the longest extinction sequence per scenario was used as the target length and shorter sequences were padded with zeros to match the target length.
\section{Results}
There were a number of extinction simulations that were exited prematurely because of improbable interactions. In these cases the algorithm failed to find alternatives to species with improbable interactions and skipped iterations three times (see section \ref{subsec:extc_alg} step \ref{itm:failsafe}). Simulations using abundance based rewiring, showed the highest percentage of aborted runs. Furthermore, especially aTl, atL, and atl networks resulted in frequent termination. Removing species from the lower level generally resulted in lower aborted simulations than removing species from the higher level. See table \ref{tab:abort_perc} in the supplemental materials for an overview. \paragraph{}
% Simulation aborts on lower level
%\begin{table}[H]
%\centering
%\caption{Number of aborted extinction simulations with initial extinction on the lower trophic level. The total number of simulation for each combination of community variable importance and rewiring method was 10000}
%\label{tab:abort_lower}
%\resizebox{\textwidth}{!}{%
%\begin{tabular}{lllllll}
% & org & Atl & aTl & atL & atl & ATL  \\ \hline
%No rewiring & 173  & 139 & 540 & 781 & 999 & 0 \\
%Abundance & 331 & 134 & 1252 & 1391 & 1561 & 0  \\
%Trait & 260 & 185 & 256 & 645 & 770 & 0  \\
%Phylogeny & 12 & 14 & 209 & 281 & 341 & 0  \\
%\begin{tabular}[c]{@{}l@{}}Abundance x \\ Trait\end{tabular} & 295 & 18 & 502 & 497 & 534 & 0  \\
%\begin{tabular}[c]{@{}l@{}}Abundance x \\ Phylogeny\end{tabular} & 69 & 151 & 537 & 775 & 850 & 0 
%\end{tabular}
%}
%\end{table}



% Simulation aborts on higher level
%\begin{table}[]
%\centering
%\caption{Number of aborted extinction simulations with initial extinction on the higher trophic level. The total number of simulation for each combination of community variable importance and rewiring method was 10000}
%\label{tab:abort_higher}
%\resizebox{\textwidth}{!}{%
%\begin{tabular}{lllllll}
% & org & Atl & aTl & atL & atl & ATL \\ \hline
%No rewiring & 245 & 189 & 752 & 1098 & 1367 & 2 \\
%Abundance & 475 & 174 & 1556 & 1864 & 2008 & 1 \\
%Trait & 315 & 253 & 339 & 886 & 1017 & 2 \\
%Phylogeny & 22 & 19 & 268 & 328 & 413 & 1 \\
%\begin{tabular}[c]{@{}l@{}}Abundance x \\ Trait\end{tabular} & 445 & 226 & 666 & 1112 & 1130 & 3 \\
%\begin{tabular}[c]{@{}l@{}}Abundance x \\ Phylogeny\end{tabular} & 118 & 36 & 637 & 694 & 725 & 0
%\end{tabular}%
%}
%\end{table}

%% in percent
%\begin{table}[H]
%\centering
%\caption{Percentage of aborted extinction simulations with initial extinction on the higher trophic level. The total number of simulation for each combination of community variable importance and rewiring method was 10000}
%\label{tab:abort_higher_perc}
%\resizebox{\textwidth}{!}{%
%\begin{tabular}{lllllll}
% & org & Atl & aTl & atL & atl & ATL  \\ \hline
%No rewiring & 2.45 & 1.89 & 7.52 & 10.98 & 13.67 & 0.02  \\
%Abundance & 4.75 & 1.74 & 15.56 & 18.64 & 20.08 & 0.01  \\
%Trait & 3.15 & 2.53 & 3.39 & 8.86 & 10.17 & 0.02  \\
%Phylogeny & 0.22 & 0.19 & 2.68 & 3.28 & 4.13 & 0.01  \\
%\begin{tabular}[c]{@{}l@{}}Abundance x \\ Trait\end{tabular} & 4.45 & 2.26 & 6.66 & 11.12 & 11.30 & 0.03  \\
%\begin{tabular}[c]{@{}l@{}}Abundance x \\ Phylogeny\end{tabular} & 1.18 & 0.36 & 6.37 & 6.94 & 7.25 & 0 
%\end{tabular}%
%}
%\end{table}

% Anova
% all
The ANOVA of the whole dataset showed, that changing the importance of community variables to simulate species interactions explained the most variance ($ R^2_{com vars} = 43.35 \% $). The number of species in the lower trophic level explained 5.14 \% of variance. The effects of the other predictors were explaining less than 1 \% of total variance each (see table \ref{tab:anova} in the supplemental material for exact values).

% lower & higher
When an ANOVA was calculated separately for removing species from the lower trophic level and from the higher trophic level, community variable importance and number of species in the lower level were still explaining the most variance. However, the number of species in the higher level had a substantially higher effect when species were removed from the higher trophic level during extinction simulation. Unexplained variance did not differ much between species removal scenarios ($ R^2_{Residuals, lower} = 49.55 \% $ and $ R^2_{Residuals, higher} = 48.04 \% $, see table \ref{tab:anova} in the supplemental materials for all R\textsuperscript{2} values). 


% com_vars w/ norew
%To inquire information about the effects of community variable importance scenarios the data was partitioned into subsets according to the 
%
% rew w/ org

\subsection{Question 1}
% Q1 comparing comvars

Changing the driving force of species interactions during the network simulation, had an effect on the general specialisation of species in a network (see figure \ref{fig:h2}). The networks that were simulated using the standard tapnet approach showed relatively high niche overlap ($ \overline{H2'}_{original} = 0.23 $). The ATL networks were most conform($ \overline{H2'}_{ATL} = 0.14 $). The only other network simulation scenario where specialisation was lower compared to the original networks was aTl ($ \overline{H2'}_{aTl} = 0.21 $). In Atl, atL, and atl networks species' interactions were more unique which led to higher H2' values ($ \overline{H2'}_{Atl} = 0.35 $, $ \overline{H2'}_{atL} = 0.28 $, and $ \overline{H2'}_{atl} = 0.37 $, respectively).
\begin{figure}[h]
	 \includegraphics[width = \textwidth]{/home/leon/Documents/Uni/M.sc/Master Thesis/Networks/models/plot_sink/h2'_all}
	 \caption{\textbf{Two dimensional Shannon Entropy of all networks} The boxplots show the H2' values for each community variable importance scenario. See section \ref{sec:net_sim} for a detailed explanation of the abbreviations of community variable importances.}
	 \label{fig:h2}
\end{figure}

To analyse the effect of community variable importance on robustness, networks were compared using no rewiring in the extinction simulation. There was a clear trend of ATL networks being most resistant to extinction cascades, regardless from which trophic level species were removed (see plots \ref{fig:extc_cv_norew_lower} and \ref{fig:extc_cv_norew_higher}). All networks showed a relatively low rate of secondary extinctions until a certain percentage of primary extinctions was met, after which secondary extinctions substanitally increased. This effect was especially present in original, Atl, and ATL networks, when species were removed from the lower trophic level.  Between 99.7 \% (ATL) and 98.1 \% (Atl) of species from the higher trophic level remained after the loss of 75 \% of species from the lower trophic level. However, this trend was less pronounced in the other networks for both species removal scenarios. \par

% compare no of secondary extinctions between lower and higher for each extinction step

It should be noted though, that the extinction curves in plots \ref{fig:extc_cv_norew_lower} and \ref{fig:extc_cv_norew_higher} only represent one network. The boxplots in figure \ref{fig:auc_cv_norew} show the robustness values of all networks and their individual extinction simulation runs.
\begin{figure}[H]
	 \includegraphics[width = \textwidth]{/home/leon/Documents/Uni/M.sc/Master Thesis/Networks/models/plot_sink/extc_sims_cv_norew_lower}
	 \caption{\textbf{Extinction cascades of a single network} The plots shows extinction curves that were simulated for one of the generated networks. Species were removed randomly from the \textbf{lower level} and no rewiring was allowed. Individual extinction simulations are represented by the black lines, while the red line shows their mean. Plot titles indicate the importances of community variables used in the network simulation process. See section \ref{sec:net_sim} for a detailed explanation of the abbreviations of community variable importances.}
	 \label{fig:extc_cv_norew_lower}
\end{figure}


\begin{figure}[H]
	 \includegraphics[width = \textwidth]{/home/leon/Documents/Uni/M.sc/Master Thesis/Networks/models/plot_sink/extc_sims_cv_norew_higher}
	 \caption{\textbf{Variation of extinction cascades of a single network} The plots shows extinction curves that were simulated for one of the generated networks. Species were removed randomly from the \textbf{higher level} and no rewiring was allowed. Individual extinction simulations are represented by the black lines, while the red line shows their mean. Plot titles indicate the importances of community variables used in the network simulation process. See section \ref{sec:net_sim} for a detailed explanation of the abbreviations of community variable importances.}
	 \label{fig:extc_cv_norew_higher}
\end{figure}

There was a clear effect of the importance of community variables during the network simulation regarding robustness. The highest robustness was observed in networks were all community variables were considered important ($\overline{R}_{ATL, lower} = 93.19$ \& $\overline{R}_{ATL, higher} = 94.25$). Networks where the importance of all community variables was set to low, showed the lowest robustness ($\overline{R}_{atl, lower} = 77.45$ \& $\overline{R}_{atl, higher} = 78.67$). Generally, networks were less affected by primary extinctions when species were removed from the higher trophic level compared to species removal from the lower level ($\overline{R}_{lower} = 83.71 $ and $\overline{R}_{higher} = 85.06$). \par


\begin{figure}[H]
	 \includegraphics[width = \textwidth]{/home/leon/Documents/Uni/M.sc/Master Thesis/Networks/models/plot_sink/auc_com_vars_by_norew}
	 \caption{\textbf{Network robustness } The boxplots show the robustness values of all extinction simulations without rewiring and random removal of species. See section \ref{sec:net_sim} for a detailed explanation of the abbreviations of community variable importances.}
	 \label{fig:auc_cv_norew}
\end{figure}



\subsection{Question 2}
% Q2 comparing rewiring

The second question focused on the differences between the rewiring methods. Therefore, only networks with the original importances of community variables were used. The majority of species from the trophic level where secondary extinctions were observed, remained in the network even after a severe loss of their interaction partners from the other trophic level. When rewiring methods were combined they only had a small effect on network robustness.
\\ Focusing on the loss of species in the final extinction step, there was little variance between the observed rewiring methods. Between the species removal scenarios, however, was a substantial difference. When animals were removed the final loss of species was considerably smaller than under plant removal. \par

% smallest last extc step lower: org_norew = 37.78, org_abund = 28.89, org_trait = 33.33, org_phylo = 24.44, org_AT = 35.55, org_AP = 40.0
% smallest last extc step higher: org_norew = 37.78, org_abund = 28.89, org_trait = 33.33, org_phylo = 24.44, org_AT = 35.55, org_AP = 40.0
\begin{figure}[H]
	 \includegraphics[width = \linewidth]{/home/leon/Documents/Uni/M.sc/Master Thesis/Networks/models/plot_sink/extc_sims_org_rew_lower}
	 \caption{\textbf{Extinction cascades of a single network} The plots shows extinction curves that were simulated for one of the generated networks. Species were removed randomly from the \textbf{lower level} using only original networks. Individual extinction simulations are represented by the black lines, while the red line shows their mean. Plot titles indicate the rewiring method used in the extinction simulation. See section \ref{itm:rew} for a details about the rewiring methods.}
	 \label{fig:extc_org_rew_lower}
\end{figure}


\begin{figure}[H]
	 \includegraphics[width = \textwidth]{/home/leon/Documents/Uni/M.sc/Master Thesis/Networks/models/plot_sink/extc_sims_org_rew_higher}
	 \caption{\textbf{Variation of extinction cascades of a single network} The plots shows extinction curves that were simulated for one of the generated networks. Species were removed randomly from the \textbf{higher level} using only original networks. Individual extinction simulations are represented by the black lines, while the red line shows their mean. Plot titles indicate the rewiring method used in the extinction simulation. See section \ref{subsec:extc_alg} step \ref{itm:rew} for details)}
	 \label{fig:extc_org_rew_higher}
\end{figure}

Network robustness seemed not to be influenced by the rewiring method to a large extend (see figure \ref{fig:auc_org_rew}). The difference in mean robustness between the most effective (phylogeny) and least effective rewiring method (abundance) was only 1.25 when removing plants and 1.23 when removing animals. Underlining the result of the first question, robustness was higher when species from the higher trophic level were removed. Combining multiple rewiring mechanisms had an averaging effect on robustness.\par 
Interestingly, simulations using abundance and abundance combined with trait rewiring, showed lower robustness than simulations without rewiring.
% seems to be connected to percentage of terminated simulation runs -> variance in robustness between extc runs quite high, so over all mean should also depict that.


!!!!! NEED TO REDO GRAPH WITH PROPER NAMES !!!!
\begin{figure}[H]
	 \includegraphics[width = \textwidth]{/home/leon/Documents/Uni/M.sc/Master Thesis/Networks/models/plot_sink/auc_org_by_rew}
	 \caption{\textbf{Network robustness } The boxplots show the robustness values of all extinction simulations fore each rewiring method and random removal of species. See section \ref{subsec:extc_alg} step \ref{itm:rew} for details) for a detailed explanation of the abbreviations of community variable importances. See section \ref{itm:rew} for a details about the rewiring methods.}
	 \label{fig:auc_org_rew}
\end{figure}

\newpage
	\section{Discussion}
% Shortly repeat the Question(s)

% Summarize the main findings
The importance of abundance, traits and phylogeny when simulating species interactions had an effect on the general specialisation of the networks as well as their robustness when no rewiring was allowed (see figures \ref{fig:h2} and \ref{fig:auc_cv_norew}). When species interactions were mainly based on abundance or phylogeny, the interactions in these networks became more unique. This was also the case when abundance, trait and phylogenetic information was used to simulate species interactions but their influence was kept low. Under the assumption that species mainly happen to interact due to matching traits, the simulations showed that the interactions in these networks were largely redundant. This suggests that niches in these networks are overlapping. The highest niche overlap was observed when the probability of an interaction between species was artificially increased for all community variables.\paragraph{}

The propagation of extinctions through networks was lowest when networks were simulated with an amplified influence of all three community variables. Assuming that either abundance, traits or phylogeny were the main driver of species interactions, led to lower network robustness compared to the networks fitted with the default tapnet approach. Furthermore, especially trait, and phylogeny based networks showed a high variance in network robustness. Networks where the interaction probabilities of all community variables were set to be low, showed the lowest robustness. \par

On the contrary, the rewiring method used in the extinction simulation with unaltered networks only had a marginal effect on network robustness (see figure \ref{fig:auc_org_rew}). Trying to shift the interactions of an extinct species to its closest phylogenetic resemblant, showed the highest effect in inhibiting extinction cascades. When species tried to compensate lost interactions by shifting them to the most abundant species in the network, the robustness was lowest.

Furthermore, removing species from the higher level always led to higher network robustness than removing species from the lower level.
\paragraph{}

% Q1
More redundancy among interactions should, in theory, lead to a higher chance of finding an adequate new partner to compensate lost interactions and thus to higher robustness. However, networks with trait based interactions were not showing an increased robustness. This could be explained by the amount of extinction simulations that were terminated prematurely. Comparing the percentages of aborted extinction simulations (table \ref{tab:abort_perc}) with the robustness of the respective interaction simulation scenario (figure \ref{fig:auc_cv_norew}) suggests that when the percentage of terminated simulations was high, robustness tended to have a higher variance and was generally lower.\\ A factor that can also lead to low network robustness is the number of species with only one interaction partner. These singleton species are the most vulnerable to coextinction, since they have to switch their interaction partner successfully in order to survive. This means that if there are a lot of singletons in a network, even though there is a high niche overlap, robustness might still be lower than what would be expected purely from looking at network specialisation.

% Q2
The fact that Abundance and Abundance x Trait rewiring scenarios led to a lower robustness than simulations without rewiring, is counter intuitive and could not be explained by the high number of terminated simulations runs compared to simulations without rewiring (compare \ref{fig:auc_org_rew} and \ref{fig:auc_org_rew_no_term}). A possible explanation for this is, that the increase of interaction probability of species due to rewiring is masked by the variability in redrawing the networks during the extinction simulations, resulting in a higher effect on robustness. To solve this problem, the difference in interaction probabilities between simulations with and without rewiring would need to be larger.
% Some rew methods still lower even when testing 25 and 50 coext_thr. Also not due to terminated scenarios. Difference in number of extinctions also unlikely, since e.g. mean extc per step for abundance was lower than for no rew.

%is most likely explained by the high number of terminated simulation runs in these scenarios (see table \ref{tab:abort_perc}). The high variability in extinction simulations of the individual networks leads to.

The minimum loss of species in the last extinction step was considerably higher when species from the lower level were removed during the extinction simulation (compare figs. \ref{fig:extc_org_rew_lower} and \ref{fig:extc_org_rew_higher}). 
Meaning that networks can support at least two species from the lower trophic level for a longer time under higher trophic species extinction than vice versa. This can be interpreted as a greater dependence of higher trophic species on lower trophic species, and was previously reported in other studies \parencite{Bascompte2006, Schleuning2016}.

%The networks simulated with abundance as the main driver of species interactions follow the assumption of neutral theory \parencite{Vazquez2005, Vazquez2007}. It proposes that species are 

% generalisation based adaptability of species to determine no of rewired interactions individually
Setting the amount of interactions a species can reallocate during rewiring to a fixed percentage, might bias the robustness of the network, since generalists should more effectively transfer their interactions to other partners than specialists. Linking the generalisation of a species to its adaptability (see section \ref{subsec:extc_alg}, step \ref{itm:shift}) would be a possibility to remove said bias but would need experimental data to quantify the effect of generalisation on rewiring success.

VB2019 found that rewiring in pollination networks is largely due to traits. 


% connectance & connectivity vary sig. between biogeographic regions (oelsen&jordano2002) but majority of sampled networks are coming from south america (e.g. brasil), apparently little data for temperate climate (i.e. europe or north america)
%In addition to the mentioned sampling biases, there also seems to be a bias in sampling location. In the case of plant-pollinator networks, South America is clearly overrepresented. For example, half of the available data on plant-pollinator networks on the Interaction Web Database (\url{http://www.ecologia.ib.usp.br/iwdb/index.html}) come from Brazil, Argentina, Chile, Venezuela, and Ecuador. This might have influenced our common understanding of interactions networks since connectance and connectivity were shown to significantly vary between biogeographic regions \parencite{Oelsen2002}.


% see rezende2007 for phylogenetic rewiring !!
Phylogeny has been identified to play a key role in the formation of interactions in bipartite networks \parencite{Rezende2007}. The authors showed that phylogenetic relatedness was predicting the identity of species' interaction partners in 46.6 \% of networks. They also found, that the number of interactions of phylogenetically related species was similar in 39 \% of networks.

% restrict formation of new interactions during rewiring process !
The extinction algorithm did not restrain species to a maximum number of interaction partners, so that species could freely form new interactions as long as the interaction probability was high enough. This could be possible if the observed network is extremely generalised, but is highly unlikely to be observed in real-world networks, since it would imply that species could diversify their foraging behaviour almost arbitrarily at any given time \footnote{However, \citeauthor{Petanidou2008} found that species that were regarded specialists in one year, tended to be generalists in a following year}. Restricting the number of new possible partners for a species is, however, not trivial. Temporal variance in interactions was reported in many studies due to a variety of reasons (see for example \cite{Olesen2008, CaraDonna2017, Schwarz2021}).

In VB when multiple attempts with multiple partners are allowed, number of attempts is the number on interactions obsevred between the species and the lost partner.

Baumgartner restricted NUMBER of interactions between species to original degree of species.
%Further complicating the issue is the fact that, while species and their interactions in a network are changing between years, the structure of the networks is relatively stable \parencite{}.


% petanidou: species are specialists in one year and generalists in another year -> might support that species are very flexible in the number of interaction partners
good approach to simulate extinction cascades over many years, since redrawing networks after an extinction can be a proxy for interannual variability in species interactions and species can be specialists in one year and generalists in another year (reason for not restricting number of new interaction partners, see petanidou2008). would additionally need some form of species turnover, e.g. by randomly drawing from a pool of species each extinction iteration.

% construct validity !! can results be translated from simulations into reality ? does it make sense to assume these connections/interrelationships/results in real networks ?
Finally, it should be considered how well the results of this analysis can be translated to real-world networks. While there is ample support of species interactions being driven by neutral and niche processes \parencite{Jordano2003, Rezende2007, Vazquez2007, Bluethgen2008}, it is unlikely that interactions are solely formed by either, abundance, traits or phylogeny as was simulated here. However, since the main goal of this study was to investigate underlying
connections of species interaction drivers and rewiring method, and their effects on network robustness, networks don't have to resemble real-world communities.

% only "quantitative" change in com_vars, could be interesting to change "qualitatively" i.e. abundance distribution (log-normal, normal, etc.), phylogeny (branching-early / branching late)...
\section{Conclusion}
Thus, while being a powerful tool, simulating extinctions with artificial networks should be done with great caution. Users should meticulously think about what the drivers of interactions in the particular networks might be and how rewiring might be restricted. Nonetheless, this approach delivered information regarding the influence of abundance, traits, and phylogeny on the formation of species interactions and their role in finding new interaction partners, by untangling their effect on network robustness.
\section{Acknowledgments}
\newpage
\section*{Declaration of Originality}
I, Leon Thoma hereby declare that the presented thesis is my own work and that I have not sought or used inadmissible help of third parties to produce this work. Furthermore, I have clearly referenced all sources used in the work and used inverted commas for all text directly or indirectly quoted from a source.\paragraph{}
This work has not been submitted to another examination institution - neither in Germany nor outside Germany - neither in the same nor in a similar way and has not been published.\paragraph{}

Freiburg,\paragraph{}

\rule{5cm}{.4pt}\par
Leon Thoma
\newpage
\section{Supplemental materials}
% in percent
\begin{landscape}
\begin{table}[H]
\centering
\captionsetup{width = .7\linewidth}
\caption{Percentage of aborted extinction simulations with initial extinction on the lower trophic level. The total number of simulation for each combination of community variable importance and rewiring method was 10000}
\label{tab:abort_perc}
\begin{tabularx}{.7\linewidth}{lllllllllllll}
\toprule
  & \multicolumn{6}{c}{Lower} & \multicolumn{6}{c}{Higher} \\ \cmidrule(l){2-7} \cmidrule(l){8-13}
 & org & Atl & aTl & atL & atl & ATL & org & Atl & aTl & atL & atl & ATL \\ \midrule
No rewiring & 1.73 & 1.39 & 5.40 & 7.81 & 9.99 & 0 & 2.45 & 1.89 & 7.52 & 10.98 & 13.67 & 0.02  \\
Abundance & 3.31 & 1.34 & 12.51 & 13.91 & 15.61 & 0 & 4.75 & 1.74 & 15.56 & 18.64 & 20.08 & 0.01 \\
Trait & 2.60 & 1.85 & 2.56 & 6.45 & 7.70 & 0 & 3.15 & 2.53 & 3.39 & 8.86 & 10.17 & 0.02 \\
Phylogeny & 0.12 & 0.14 & 2.09 & 2.81 & 3.41 & 0 & 0.22 & 0.19 & 2.68 & 3.28 & 4.13 & 0.01 \\
Abundance x \\ Trait & 2.95 & 0.18 & 5.02 & 4.97 & 5.34 & 0 & 4.45 & 2.26 & 6.66 & 11.12 & 11.30 & 0.03 \\
Abundance x \\ Phylogeny & 0.69 & 1.51 & 5.37 & 7.75 & 8.50 & 0 & 1.18 & 0.36 & 6.37 & 6.94 & 7.25 & 0 \\ \bottomrule
\end{tabularx}%
\end{table}
\end{landscape}
%($R^2_{N_higher} = 0.67 %$, $R^2_{Trophic_level} = 0.55 %$, $R^2_{Rewiring} = 0.34 %$, $R^2_{H2'} = 0.33 %$, $R^2_{Rewiring x H2'} = 0.05%$)


\begin{landscape}
\begin{table}
\label{tab:anova}
\caption{ANOVA table. The first line of the table states the subset of data used. 'All' refers to the whole dataset, 'Lower' to simulations where species were removed from the lower trophic level and 'Higher' to simulations were species were removed from the higher trophic level. N\textsubscript{lower} and N\textsubscript{higher} are the number of species in the respective trophic level, Community variables are the species interaction scenarios as described in section \ref{sec:net_sim}, Rewiring methods refer to the rewiring methods explained in section \ref{subsec:extc_alg} step \ref{itm:rew}, $H2'$ is the two dimensional shannon entropy \parencite{Bluethgen2006}, and Rewiring method x $H2'$ is the interaction between rewiring methods and $H2'$}
\begin{tabularx}{\linewidth}{@{} X *9{c} @{}}
\toprule
  & \multicolumn{3}{c}{All} & \multicolumn{3}{c}{Lower} & \multicolumn{3}{c}{Higher} \\ \cmidrule(l){2-4} \cmidrule(l){5-7} \cmidrule(l){8-10}
  & Df & Sum of squares & $R^2$ & Df & Sum of squares & $R^2$ & Df & Sum of squares & $R^2$  \\ \midrule
Residuals & 679264 & 18931283 & 49.54 \% & 343267 & 9737201 & 49.55 \% & 335979 & 8799947 & 48.04 \%\\ 
N\textsubscript{lower} & 1 & 1966173 & 5.14 \% & 1 & 1412044 & 7.19 \% & 1 & 622401 & 3.40 \%\\
N\textsubscript{higher} & 1 & 258186 & 0.67 \% & 1 & 424 & 0.002 \% & 1 & 555864 & 3.03 \%\\
Community variables & 5 & 16565231 & 43.35 \% & 5 & 8344401 & 42.46 \% & 5 & 8219474 & 44.87 \%\\
Rewiring method & 5 & 131960 & 0.34 \% & 5 & 69972 & 0.36 \% & 5 & 62030 & 0.34 \%\\ 
$H2'$ & 1 & 126507 & 0.33 \% & 1 & 76790 & 0.39 \% & 1 & 48687 & 0.27 \%\\
Rewiring method x $H2'$ & 5 & 20551 & 0.05 \% & 5 & 10400 & 0.05 \% & 5 & 9965 & 0.05 \%\\ 
Trophic level & 1 & 213951 & 0.55 \%  \\ \bottomrule
\end{tabularx}
\end{table}
\end{landscape}



\begin{multicols}{2}[\printbibheading]
\printbibliography[heading=none]
\end{multicols}
\end{document}
